%% Uncomment one of these:
%% the 1st when using pdflatex, which directly typesets your document in
%% pdf (use jpg or pdf figures), or
%% the 2nd when producing a ps \frac{?}{?}ile (use eps figures, don't use ps figures!).
\documentclass[english,12pt,a4paper,pdftex,elec,utf8]{aaltothesis}

%include used packages
\usepackage{verbatim}
\usepackage{graphicx}
\usepackage{colortbl}
\usepackage{amsfonts,amssymb,amsbsy}
\usepackage[parfill]{parskip}
\usepackage{cite}
\usepackage{libertine} 
\usepackage{url}
\usepackage{multirow}
\usepackage[hidelinks]{hyperref}
\usepackage [final] {pdfpages}
\usepackage{soul}%used for highlighting command
\usepackage{pdfpages} %forpdfappendices
\usepackage{float}
\usepackage[version=4]{mhchem} % Mikael added this for typesetting of IMC names like Cu3Sn
                    % usage: \ce{Cu3Sn}
                    
\usepackage{textgreek} % Mikael added this for typesetting Greek letters indicating phases inline without using math mode (as it changes to italic).
  
                      % usage: \textbeta, \textgamma

\usepackage{array}
\newcolumntype{C}[1]{>{\centering\arraybackslash}p{#1}}
\hypersetup{
  %linktocpage,
  pdfpagemode=UseNone,
  pdfstartview=FitH,
  linktoc=all,
  colorlinks = true,
  urlcolor=black,
  linkcolor=black,
  citecolor=black,
  pdftitle={Low Temperature SLID Bond for MOEMS Components},
  pdfauthor={Joseph Hotchkiss},
  pdfkeywords={Materials Engineering}
  }

\usepackage{gensymb}
%custom color boxes with text wrapping 



%%%%%%%%%%%%%%%%%%%%   Document Starts Here  %%%%%%%%%%%%%%%%%%%%%%%%%%%%

\begin{document}

%%%%%%%%%%%%%%%%%%%%%    coverpage data   %%%%%%%%%%%%%%%%%%%%%%%%%%%%%%
%%%%%%%%%%%%%%%%%%%%%%%%%%%%%%%%%%%%%%%%%%%%%%%%%%%%%%%%%%%%%
%%%%%%%%%%%%%%%%%%%%%%%%%%%%%%%%%%%%%%%%%%%%%%%%%%%%%%%%%%%%%

%\renewcommand{\thesissupervisorname}{Instructor}
\department{Department of Electrical Engineering and Automation}
\univdegree{MSc}
\author{Jan von Steuben 661119 \\ Joseph Hotchkiss 658656}
\thesistitle{Translational Engineering Forum: Overview Report 1}
\place{Espoo}
\date{11.10.2018}
%\supervisor{Glenn Ross} 
\uselogo{aaltoBlue}{?}

%%%%%%%%%%%%%%%%%%%%%%%%%%%%%%%%%%%%%%%%%%%%%%%%%%%%%%%%%%%%%

%%%%%%%%%%%%%%%%%%%% Create the coverpage  %%%%%%%%%%%%%%%%%%%%%%%%%%%%%
\makecoverpage
%%%%%%%%%%%%%%%%%%%%%%%%%%%%%%%%%%%%%%%%%%%%%%%%%%%%%%%%%%%%%

\newpage

%%%%%%%%%%%%%%%%%%%%%% Table of contents  %%%%%%%%%%%%%%%%%%%%%%%%%%%%%

\cleardoublepage
\storeinipagenumber
\pagenumbering{arabic}
\setcounter{page}{1}
\thispagestyle{empty}

%%%%%%%%%%%%%%%%%%%%%%%%%%%%%%%%%%%%%%%%%%%%%%%%%%%%%%%%%%%%

%%%%%%%%%%%%%%%%%%%%%%%  Document Body  %%%%%%%%%%%%%%%%%%%%%%%%%%%%
\section*{Introduction}

	Translational Engineering is a multi-disciplinary branch of engineering by which the gap between purely academic research and industry is bridged. It is engineering with the goal of innovating a product. The first three exercises of the Translational Engineering Forum course dealt with challenges one might face when starting Translational Engineering research and methods to help overcome them. The first exercise tried to bring forth the understanding that when one ventures into this type of project, they will need to find support from experts in the field of their engineering research.  

	It is essential at the start of a project to identify all of the areas that you will be working in and to find suitable people to help to point you in the right direction. The first exercise dealt with the design challenges of a smart, wireless glucose meter. Suitable people were found from Aalto University, that could provide either valuable information about the fields of interest, or perhaps help to find a more suitable person. It could be understood that once you have a support group in order, that you would start to look at technologies that will be required in your project. 
	
	Exercise two was there to help show how many challenges one might face when trying to develop a technical project. The task was to come up with a method to monitor the health of a bridge. The challenge being that you want to make a viable product in the end. This task should be able to be inexpensively integrated into new bridge constructions as well as retrofitted into existing structures. This exercise helps to show one of the main challenges of Translational engineering; making a desirable product that is still convenient and affordable. If this challenge can be overcome, then one should consider trying to market and sell their product. 
	
	Exercise three dealt with getting funding and managing the beginning years of such a venture. When a person has made such an innovation and they have fanned off into a startup company, they will need to get funding from somewhere in order to grow their innovation into a product. This is a long and arduous process that will require a lot of work and knowhow. If a person or team can maneuver through all of these hurdles it is possible that they will be able to be successful with their innovation. 
	
	The first three exercises attempted to give a gross overview of translational engineering from conception to end product. The challenge is great as it is a field in which these projects take years to get to any of these points, and one will not succeed without considerable help and knowledge. 
	


\newpage


\section*{Exercise 1}

	In the first exercise, we looked at a pancreas pump, and how it could be used to support the patient in the best way. This included data transfer to the cloud and access by the doctor. As the development of the whole chain is difficult and needs a variety of different skills, networking is required to seek out the experts to support in gaining the right understanding. The exercise itself was aimed at realizing that designing a product, system or service is not that simple, and many different competences are required. We then started looking at who at Aalto would be the specialists, who do have the knowledge in the related areas of expertise or can provide us some information who might know. The comprehensive list can be found from exercise one. 

	Our idea was to take the artificial pancreas sensor and use the data for controlling the pump. In addition, sending the data wirelessly to a phone. Skills required are in sensors, embedded hardware and software, as well as regulations in relation to medical/implantable devices. This also requires expertise in material compatibility. Furthermore, wireless expertise, will be required. With that comes information security, both for the wireless link as well as data storage, as it is very personal, and in case of a 2-way communication, a hacking attack could even lead to death by operating the pump wrongly. For the wireless transmission, short-range radio transmissions are used for several reasons. One is that the amount of data to be sent is small, in addition to allow for a good battery life. Furthermore, as the sensor is implanted, there are regulations to the amount and level of RF radiation that is allowed. 

	After transferring the data to the cloud, it can be accessed by the doctor to be able to perform a diagnose remotely and maybe adjust the settings to the pump for an optimized treatment. This remote access would allow a fast reaction in case of troubles to improve patient safety. The user interface design is another challenge, as too much information as well as too little or the wrong information will render the system useless and / or complicated. The information visible to the doctor should be the critical ones by default, and the chance to enable more if the need occurs. 
	
	For a simple thing like the pancreas pump, there is currently 13 experts to consult for the system, and this does not yet include experts related to manufacturing.  


\newpage


\section*{Exercise 2}

	In the second exercise the task was to find a suitable system to monitor bridge health. The system was to be inexpensive and capable of being integrated into existing bridges as well as new construction. The challenge of this exercise is compounded by having very little understanding of the stressors of a bridge as well as potential weak points. Imagination and existing knowledge was used to determine what things to consider.
	
	The aspects considered were useful sensors, lifetime of the system, cost, power considerations, communication, and integration into the bridge. It was determined that since cost is relative that high quality sensors should be used wherever possible. This is compounded by the fact that the bridge monitor is also a safety critical application and that having a robust reliable system would reduce the need for man power to investigate potential issues. The use of high quality components in the system should also increase its lifetime. 
	
	As far as power considerations, it was deemed that since the bridge is likely connected to the power grid, then in most cases the system could be powered directly from the power network. In other cases the system could be made battery operated and energy consumption could be minimized by taking less frequent measurements and only transmitting critical data. As for communication, as with all safety critical applications a wired connection would be ideal. However, several radio protocols and even an RFID system was considered and in theory could be implemented. 
	
	The largest challenge still comes from the integration of the system into the bridge. The environment is harsh and it is even more harsh if sensors were to be embedded into wet concrete. These things were considered and considerations and compromises would be required in order to make the system suitable for new and old bridges. The conclusion from the exercise was that, perhaps the best solution would be to have a system that still required human inspection when a large or sudden change in the bridge structure was detected. This would alleviate some costs over time, but require a somewhat large one time investment to purchase and install  the system. 
	
	


\newpage


\section*{Exercise 3}

	When starting a business, there are different ways of financing. The level of money required depends largely on the product or service being developed, and at the stage of development. In a software only business, the initial costs are minimal, as in practice all that is needed is a computer and nowadays that can be found in any household. When designing a tangible product, more money might be required for prototypes, tests and even evaluation of the idea. 

	For any kind of external money acquisition, a clear business plan must be present. When financing from one?s own pocket, or borrowing/involving family and friends, this must not be true, but in general, everyone putting time and money into an idea will want to see a plan on how to succeed. Using of crowdfunding can be an easy way to get some additional cash and an evaluation of the idea, as people will not invest in an idea they do not like or see a business. Going to a bank or investor, a clear strategy will often be required. It is of course possible that they will purchase the technology especially in case of intellectual property. 

	There are also governmental or European Union funding (e.g. Horizon 2020) possible. There are funds for research, and others for start-ups for example from Business Finland. Moreover, joint ventures between industry and research institutes can be beneficial, of course depending on the type of product or service provided. Funding is often quite limited and other options for obtaining money might have to be considered. Using of e.g. venture capitalists, they will spend their money and time/expertise in exchange for equity. On the other hand, with their support and connections, a business can take off quite fast. One thing must be kept in mind when looking for investors or partners, the strategies must be matching to avoid conflicts. Some of the investors are interested only in growing the business fast and then sell it for a lot of profit. If your business is your life, you might not want to part with it, even if it means to decline an offer.  




\end{document}



