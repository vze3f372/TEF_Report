%% Uncomment one of these:
%% the 1st when using pdflatex, which directly typesets your document in
%% pdf (use jpg or pdf figures), or
%% the 2nd when producing a ps \frac{?}{?}ile (use eps figures, don't use ps figures!).
\documentclass[english,12pt,a4paper,pdftex,elec,utf8]{aaltothesis}

%include used packages
\usepackage{verbatim}
\usepackage{graphicx}
\usepackage{colortbl}
\usepackage{amsfonts,amssymb,amsbsy}
\usepackage[parfill]{parskip}
\usepackage{cite}
\usepackage{libertine} 
\usepackage{url}
\usepackage{multirow}
\usepackage[hidelinks]{hyperref}
\usepackage [final] {pdfpages}
\usepackage{soul}%used for highlighting command
\usepackage{pdfpages} %forpdfappendices
\usepackage{float}
\usepackage[version=4]{mhchem} % Mikael added this for typesetting of IMC names like Cu3Sn
                    % usage: \ce{Cu3Sn}
                    
\usepackage{textgreek} % Mikael added this for typesetting Greek letters indicating phases inline without using math mode (as it changes to italic).
  
                      % usage: \textbeta, \textgamma

\usepackage{array}
\newcolumntype{C}[1]{>{\centering\arraybackslash}p{#1}}
\hypersetup{
  %linktocpage,
  pdfpagemode=UseNone,
  pdfstartview=FitH,
  linktoc=all,
  colorlinks = true,
  urlcolor=black,
  linkcolor=black,
  citecolor=black,
  pdftitle={Low Temperature SLID Bond for MOEMS Components},
  pdfauthor={Joseph Hotchkiss},
  pdfkeywords={Materials Engineering}
  }

\usepackage{gensymb}
%custom color boxes with text wrapping 



%%%%%%%%%%%%%%%%%%%%   Document Starts Here  %%%%%%%%%%%%%%%%%%%%%%%%%%%%

\begin{document}

%%%%%%%%%%%%%%%%%%%%%    coverpage data   %%%%%%%%%%%%%%%%%%%%%%%%%%%%%%
%%%%%%%%%%%%%%%%%%%%%%%%%%%%%%%%%%%%%%%%%%%%%%%%%%%%%%%%%%%%%
%%%%%%%%%%%%%%%%%%%%%%%%%%%%%%%%%%%%%%%%%%%%%%%%%%%%%%%%%%%%%

%\renewcommand{\thesissupervisorname}{Instructor}
\department{Department of Electrical Engineering and Automation}
\univdegree{MSc}
\author{Jan von Steuben 661119\\ Joseph Hotchkiss 658656}
\thesistitle{Translational Engineering Forum: Overview Report 1}
\place{Espoo}
\date{11.10.2018}
%\supervisor{Glenn Ross} 
\uselogo{aaltoBlue}{?}

%%%%%%%%%%%%%%%%%%%%%%%%%%%%%%%%%%%%%%%%%%%%%%%%%%%%%%%%%%%%%

%%%%%%%%%%%%%%%%%%%% Create the coverpage  %%%%%%%%%%%%%%%%%%%%%%%%%%%%%
\makecoverpage
%%%%%%%%%%%%%%%%%%%%%%%%%%%%%%%%%%%%%%%%%%%%%%%%%%%%%%%%%%%%%

\iffalse
\keywords{SLID, Intermetallic Compounds, Tin, Indium, Copper, Bond, Temperature }
\begin{abstractpage}[english]
    
    Solid Liquid Interdiffusion is an interesting bonding solution for its potential to create strong and reliable bonds at lower temperatures that other traditional methods, while at the same time the resultant bonds are able to withstand higher temperatures experienced during additional processing steps. This is especially interesting for MOEMS devices, as it enables the miniaturization of said components by limiting the negative effects of thermal expansions between different materials that may be needed to be bound together. \\
    
    This project looked at prospective material combinations to form a SLID bond at temperatures below 300\celsius~and form intermetallic compounds that will withstand additional processing  temperatures in excess of 450\celsius. The materials that were selected for this project were copper, tin and indium. Silicon wafers were processed and bonded in Aalto's Micronova facility. Bonds were achieved despite difficulties in the deposition of indium. Bonded samples were diced and analyzed for inspection of bond quality and composition.\\ 
    
    Chip level bonding was successfully at 250\celsius~and were analyzed by using both optical and scanning electron microscopy. In samples bonded for 2500~seconds, the low temperature material appeared to dissolve the high temperature material as expected and the formation of intermetallics was evident. Line and point scans showed lower amounts of indium in the system than expected. However the effects of indium on the system are as of yet unknown and will be studied further in the future, as will improving the indium plating process at Aalto's Micronova facilities. 
    
       
    
    


\end{abstractpage}
\fi
%% Make sure that the ToC begins on a new page

\newpage

%%%%%%%%%%%%%%%%%%%%%% Table of contents  %%%%%%%%%%%%%%%%%%%%%%%%%%%%%
\iffalse
\thesistableofcontents

%%%%%%%%%%%%%%%%%%%%%%%%%%%%%%%%%%%%%%%%%%%%%%%%%%%%%%%%%%%%%


%%%%%%%%%%%%%%%%%%%% Symbols and abbreviations  %%%%%%%%%%%%%%%%%%%%%%%%%%%
\mysection{Abbreviations}
	%\subsection*{Abbreviations}

		\begin{tabular}{ll}
                            
            SLID & Solid-Liquid Interdiffusion \\
            MEMS &  Micro-Electro-Mechanical Systems \\ 
            NEMS &  Nano-Electro-Mechanical Systems \\
            MOEMS & Micro-Opto-Electro-Mechanical Systems \\ 
            LIDAR & Light Detection and Ranging \\                          
            CTE & Coefficient of Thermal Expansion \\
            SEM & Scanning Electron Microscope \\ 
            IMC & Intermetallic Compound \\
            EDS & Energy-dispersive X-ray spectroscopy \\
            SEM & Scanning electron microscope \\
            HMDS & Hexamethyldisilazane \\

		\end{tabular}

%%%%%%%%%%%%%%%%%%%%%%%%%%%%%%%%%%%%%%%%%%%%%%%%%%%%%%%%%%%%

%%%%%%%%%%%%%%%%%%%%%%%%  Page Setup  %%%%%%%%%%%%%%%%%%%%%%%%%%%%%

%% Tweaks the page numbering to meet the requirement of the thesis format:
%% Begin the page numbering in Arabian numerals (and leave the first page
%% of the text body empty, see \thispagestyle{empty} below).
%% Additionally, force the actual text to begin on a new page with the 
%% \clearpage command.
%% \clearpage is similar to \newpage, but it also flushes the floats (figures
%% and tables).
%% There is no need to change these
%%

\fi

\cleardoublepage
\storeinipagenumber
\pagenumbering{arabic}
\setcounter{page}{1}
\thispagestyle{empty}

%%%%%%%%%%%%%%%%%%%%%%%%%%%%%%%%%%%%%%%%%%%%%%%%%%%%%%%%%%%%

%%%%%%%%%%%%%%%%%%%%%%%  Document Body  %%%%%%%%%%%%%%%%%%%%%%%%%%%%
\section*{Introduction}

	Translational Engineering is a multi-disciplinary branch of engineering by which the gap between purely academic research and industry is bridged. It is engineering with the goal of innovating a product. The first three exercises of the Translational Engineering Forum course dealt with challenges one might face when starting Translational Engineering research and methods to help overcome them. The first exercise tried to bring forth the understanding that when one ventures into this type of project, they will need to find support from experts in the field of their engineering research.  

	It is essential at the start of a project to identify all of the areas that you will be working in and to find suitable people to help to point you in the right direction. The first exercise dealt with the design challenges of a smart, wireless glucose meter. Suitable people were found from Aalto University, that could provide either valuable information about the fields of interest, or perhaps help to find a more suitable person. It could be understood that once you have a support group in order, that you would start to look at technologies that will be required in your project. 
	
	Exercise two was there to help show how many challenges one might face when trying to develop a technical project. The task was to come up with a method to monitor the health of a bridge. The challenge being that you want to make a viable product in the end. This task should be able to be inexpensively integrated into new bridge constructions as well as retrofitted into existing structures. This exercise helps to show one of the main challenges of Translational engineering; making a desirable product that is still convenient and affordable. If this challenge can be overcome, then one should consider trying to market and sell their product. 
	
	Exercise three dealt with getting funding and managing the beginning years of such a venture. When a person has made such an innovation and they have fanned off into a startup company, they will need to get funding from somewhere in order to grow their innovation into a product. This is a long and arduous process that will require a lot of work and knowhow. If a person or team can maneuver through all of these hurdles it is possible that they will be able to be successful with their innovation. 
	
	The first three exercises attempted to give a gross overview of translational engineering from conception to end product. The challenge is great as it is a field in which these projects take years to get to any of these points, and one will not succeed without considerable help and knowledge. 
	


\newpage


\section*{Exercise 1}

	In the first exercise, we looked at a pancreas pump, and how it could be used to support the patient in the best way. This included data transfer to the cloud and access by the doctor. As the development of the whole chain is difficult and needs a variety of different skills, networking is required to seek out the experts to support in gaining the right understanding. The exercise itself was aimed at realizing that designing a product, system or service is not that simple, and many different competences are required. We then started looking at who at Aalto would be the specialists, who do have the knowledge in the related areas of expertise or can provide us some information who might know. The comprehensive list can be found from exercise one. 

	Our idea was to take the artificial pancreas sensor and use the data for controlling the pump. In addition, sending the data wirelessly to a phone. Skills required are in sensors, embedded hardware and software, as well as regulations in relation to medical/implantable devices. This also requires expertise in material compatibility. Furthermore, wireless expertise, will be required. With that comes information security, both for the wireless link as well as data storage, as it is very personal, and in case of a 2-way communication, a hacking attack could even lead to death by operating the pump wrongly. For the wireless transmission, short-range radio transmissions are used for several reasons. One is that the amount of data to be sent is small, in addition to allow for a good battery life. Furthermore, as the sensor is implanted, there are regulations to the amount and level of RF radiation that is allowed. 

	After transferring the data to the cloud, it can be accessed by the doctor to be able to perform a diagnose remotely and maybe adjust the settings to the pump for an optimized treatment. This remote access would allow a fast reaction in case of troubles to improve patient safety. The user interface design is another challenge, as too much information as well as too little or the wrong information will render the system useless and / or complicated. The information visible to the doctor should be the critical ones by default, and the chance to enable more if the need occurs. 
	
	For a simple thing like the pancreas pump, there is currently 13 experts to consult for the system, and this does not yet include experts related to manufacturing.  


\newpage


\section*{Exercise 2}

	In the second exercise the task was to find a suitable system to monitor bridge health. The system was to be inexpensive and capable of being integrated into existing bridges as well as new construction. The challenge of this exercise is compounded by having very little understanding of the stressors of a bridge as well as potential weak points. Imagination and existing knowledge was used to determine what things to consider.
	
	The aspects considered were useful sensors, lifetime of the system, cost, power considerations, communication, and integration into the bridge. It was determined that since cost is relative that high quality sensors should be used wherever possible. This is compounded by the fact that the bridge monitor is also a safety critical application and that having a robust reliable system would reduce the need for man power to investigate potential issues. The use of high quality components in the system should also increase its lifetime. 
	
	As far as power considerations, it was deemed that since the bridge is likely connected to the power grid, then in most cases the system could be powered directly from the power network. In other cases the system could be made battery operated and energy consumption could be minimized by taking less frequent measurements and only transmitting critical data. As for communication, as with all safety critical applications a wired connection would be ideal. However, several radio protocols and even an RFID system was considered and in theory could be implemented. 
	
	The largest challenge still comes from the integration of the system into the bridge. The environment is harsh and it is even more harsh if sensors were to be embedded into wet concrete. These things were considered and considerations and compromises would be required in order to make the system suitable for new and old bridges. The conclusion from the exercise was that, perhaps the best solution would be to have a system that still required human inspection when a large or sudden change in the bridge structure was detected. This would alleviate some costs over time, but require a somewhat large one time investment to purchase and install  the system. 
	
	


\newpage


\section*{Exercise 3}

	When starting a business, there are different ways of financing. The level of money required depends largely on the product or service being developed, and at the stage of development. In a software only business, the initial costs are minimal, as in practice all that is needed is a computer and nowadays that can be found in any household. When designing a tangible product, more money might be required for prototypes, tests and even evaluation of the idea. 

	For any kind of external money acquisition, a clear business plan must be present. When financing from one?s own pocket, or borrowing/involving family and friends, this must not be true, but in general, everyone putting time and money into an idea will want to see a plan on how to succeed. Using of crowdfunding can be an easy way to get some additional cash and an evaluation of the idea, as people will not invest in an idea they do not like or see a business. Going to a bank or investor, a clear strategy will often be required. It is of course possible that they will purchase the technology especially in case of intellectual property. 

	There are also governmental or European Union funding (e.g. Horizon 2020) possible. There are funds for research, and others for start-ups for example from Business Finland. Moreover, joint ventures between industry and research institutes can be beneficial, of course depending on the type of product or service provided. 




\end{document}









































































































\iffalse

\section{Introduction}

	A low temperature bonding process is very interesting in the field of microfabrication as it would allow for an improvement in reliability and cost across a range of electronic devices. One of the main application areas is creating low-cost, highly reliable Micro-Opto-Electro-Mechanical System~(MOEMS) sensors for Light Detection and Ranging~(LIDAR) systems. The driving force behind this is the automotive industry and their desire to realize true autonomously driving vehicles, which could greatly increase passenger safety. As these sensors need a transparent cap material to allow light to pass through to the sensor element, bonding glass or some other transparent material to a silicon wafer is required. This requirement introduces some challenges as the two materials to be bonded have different coefficients of thermal expansion~(CTE) which causes greater stresses at higher temperatures~\cite{institute_of_electrical_and_electronics_engineers_2010_2010}. Traditional glass frit bonding uses a reflow based process utilizing molten glass to achieve a bond. The typical bonding temperature of this method is just below 450\celsius~\cite{lindroos_handbook_2015}. Having a Si-glass bond at these temperatures will introduce additional stress to the bond interface~\cite[p.~527]{institute_of_electrical_and_electronics_engineers_2010_2010}.
	
	The purpose of this project was to carry out material research to identify potential metal systems to carryout a Solid-Liquid Interdiffusion~(SLID) bond at temperatures below 300\celsius. Additionally bond composition and quality analysis of the bonds were carried out by means of optical and electron microscopy.  SLID bonding is a process which utilizes low melting point and high melting point metals. Optimally during SLID bonding the low temperature melting point metal becomes liquid, the high melting point metal rapidly dissolves and should create stable intermetallic compounds~(IMC) that can withstand higher temperatures than the original system~\cite{bernstein_semiconductor_1966}\cite{xu_wafer-level_2013}. Therefore the main challenge in this project was finding a material combination that would form a bond at temperatures below 300\celsius~ and form stable intermetallic compounds that could withstand additional processing steps at temperatures in excess of 400\celsius, while at the same time being able to produce these bonds in Aalto University's Micronova facilities.  
	
	Bonding experiments were carried out using different material amounts and bonding times. Once bonds were successfully created, they were then inspected with an optical and a scanning electron microscope~(SEM) for bond quality and composition. This helps to ensure that the desired intermetallic compounds~(IMC) were formed and that the bond will endure temperatures experienced during subsequent process steps. This also allowed for the detection of any voids or cracks in the interfacial structure, indicative of bond weakness.
	
	
	
%% Start next section on a new page
\clearpage

\section{Current Wafer Bonding Trends}
	
	Wafer bonding is a technology used for packaging two or more wafers together and its popularity has increased as the size of component packaging has decreased due to the birth and growing complexity of MEMS and Nano-Electro-Mechanical Systems (NEMS) devices. Bonding techniques can be divided into two main classes: Direct bonding and bonding with an intermediate layer\cite{franssila_introduction_2010}. In direct bonding the bond is achieved  without the use of adhesive materials whereby two silicon wafers are bonded directly using very high processing temperatures. The formed bond is strong and reliable as it is formed between the silicon and its native oxide. \cite{franssila_introduction_2010}. Bonding with an intermediate layer is suitable for a wider variety of applications as it requires significantly lower bonding temperatures than direct bonding methods. Typically indirect bonding techniques utilize either glass, metal, polymer, or epoxy adhesives \cite{franssila_introduction_2010}. The following subsection discusses the most commonly used indirect bonding techniques used for MEMS manufacturing today. 
	
	
\subsection {Glass frit}
        Glass frit is a widely used wafer bonding technology which utilizes an intermediate glass layer and is especially useful for encapsulation of MEMS devices \cite{universal_glass_frit}. Bonding encapsulation is a process which is carried out at the end of device production stages. Requirements for MEMS encapsulation are limited process temperatures, good hermetical seal,  sufficient mechanical strength of the bond, good process yield and reliability of the bond \cite{handbook_of_wafer_bonding}. Glass frit bonding fulfils MEMS encapsulation bonding requirements and in addition to that, it is capable of bonding almost any commonly found surface in microelectronics and microsystems fabrication process ~\cite{handbook_of_wafer_bonding}.   
    
        Glass frit technology is based on compressing low melting point glass between wafers which will then form an intermediate layer between bondable objects \cite{universal_glass_frit}. This bonding process can be divided into three main steps: the glass paste deposition, conditioning of the paste and the actual bonding. The glass paste is usually deposited by screen printing, is capable of bonding structures with high steps or holes, and is applicable for mass fabrication \cite{universal_glass_frit,screenprinting}.  During the conditioning step the glass paste is heated until its viscosity is low enough to ensure that the surface is wetted and the inequalities covered \cite{handbook_of_wafer_bonding}. Organic additives will outgas during the heating step which will compact the glass interlayer and form strong hermetic seal when cooled. Figure~\ref{attachment} shows a gyroscope bonded using a glass frit process. This sensor uses metal line for signal transferring which is typical for glass frit bond ~\cite{handbook_of_wafer_bonding}. 
   
		
		Glass frit material requirements are quite strict and only special low melting point glasses can be used. Currently utilized glass frit mixtures contain lead which is already banned from regular solders. In addition the glass frit paste usually contains filler materials to even out the CTE with silicon. Lead zinc silicate and lead borate glasses are commonly used.~\cite{handbook_of_wafer_bonding}
		
		\begin{figure}[htb]
        \begin{center}
        \includegraphics[width=15cm]{glassfritbond1.JPG}
        \end{center}
        \caption{Glass frit bonding example from MEMS capping process~\cite{handbook_of_wafer_bonding}}
        \label{attachment}
        \end{figure}
    
%	In order to allow a good reflow of the glass during bonding and to ensure sufficient bond quality,     5~\textmu m or thicker layer must be deposited \cite{handbook_of_wafer_bonding}. Minimum feature size of glass frit bonding with screen printing technology is rather large 190~\textmu m \cite{franssila_introduction_2010}. However, photomasks are not required for screen printing \cite{franssila_introduction_2010}.

	The bonding process must be carefully designed because the bond quality is sensitive to bonding temperature, alignment and pressure. 
	
	In addition to the previously mentioned advantages of glass frit bond: it has very good reliability, low stress on bonded components, safe and reproducible process and no surface activation is required~\cite{handbook_of_wafer_bonding}. 

	%Alignment errors smaller than 5~\textmu m can be achieved. 
	
		Glass frit bonding is mainly used in the encapsulation and micromachined sensors such as gyroscopes and acceleration sensors. These applications benefit from the glass frit due to such features as hermetic sealing, metallic lead-throughs and high bonding yield. In addition to encapsulation, glass frit bonding is widely used to bond fully processed cap wafers in MEMS manufacturing. However, these are just typical applications and universal technology such as glass frit has countless applications.~\cite{handbook_of_wafer_bonding}
		
		

	\subsection{Anodic bonding}
    Anodic bonding is the oldest bonding technique found in microfabrication but it has favourable properties for many applications~\cite{franssila_introduction_2010}. This is yet another glass material bonding technique which has proven to be a reliable method for packaging at wafer level~\cite{handbook_of_wafer_bonding,franssila_introduction_2010}. Currently, anodic bonding is reported to account for a majority of packaging applications in micromechanical systems~\cite{handbook_of_wafer_bonding}. This process is well established at the industry-level, provides high quality hermetic seals, does not have strict cleanliness requirements, and can tolerate rougher surfaces than other techniques~\cite{handbook_of_wafer_bonding}. However this bonding process involves temperature ramping to approximately 400\celsius, exposure to a strong electric field, and the glass used for bonding contains alkali which can create some incompatibilities devices~\cite{handbook_of_wafer_bonding}. For example, due to the voltage requirements of anodic bonding CMOS devices cannot be manufactured using this technology.  %Anodic bonding uses a relatively low processing temperature but the process requires high electric fields and the glass used for bonding contains alkali which will create some incompatibilities~\cite{handbook_of_wafer_bonding}. 
    Additionally, the anodic bonding environment does not have strict cleanliness requirements and the bonds can tolerate rougher surfaces than other bonding techniqus~\cite{handbook_of_wafer_bonding}.
    
   % Anodic bonding process relies on the polarization of alkali-containing glasses, using high DC voltage (400-1000~V) and temperature of 300-250\celsius~\cite{handbook_of_wafer_bonding}. This process separates sodium and oxygen ions in the different side of the interlayer creating electric field which pulls the parts together with high force~\cite{franssila_introduction_2010}. The process consists of the following steps. Bonding is typically initiated by applying pressure into the center of the wafer~\cite{franssila_introduction_2010}. After initialization the samples are aligned and chamber is made ready for the bonding which includes chamber evaluation and temperature ramping~\cite{handbook_of_wafer_bonding}. When the chamber and the sample are ready the bonding can be done~\cite{handbook_of_wafer_bonding}. The process typically lasts 30-60~minutes, but usually batch operation suitable equipment are used, which minimizes sample loading time~\cite{handbook_of_wafer_bonding}. Figure~\ref{attachment2} shows anodic bonding working principle.
    
    %    \begin{figure}[htb]
    %    \begin{center}
    %   \includegraphics[width=15cm]{glassfritbond2.JPG}
    %    \end{center}
    %    \caption{Example of anodic bonding working principle  \cite{franssila_introduction_2010}}
    %    \label{attachment2}
    %    \end{figure}
    
    %Anodic bonding creates strong hermetic sealing and it can conform surface roughness up to 50~nm~\cite{franssila_introduction_2010}. Anodic bonds are visually inspectable through the glass side~\cite{franssila_introduction_2010}. Anodic bonding basic processing parameters (temperature, applied voltage and time) are tied together \cite{franssila_introduction_2010}. 
    Anodic bonds can be successful even at 150 \celsius ~if the time and voltage parameters are sufficiently high.~\cite{franssila_introduction_2010}. Commonly used glass such as Pyrex has CTE close to silicon and typically will induce some stress to the bonded interfaces \cite{handbook_of_wafer_bonding}. In terms of bonding properties anodic bond and glass frit bonds are comparable. %Compared to glass frit bonding, anodic bonding has smaller seams but it does not cover metallizations and signal through performance does not compete against glass frit technology \cite{handbook_of_wafer_bonding}. 
    

	\subsection{Eutectic bonding}
	Eutectic bonding is a bonding technology which relies on metal compositions which have eutectic points. In eutectic point all three phases of solid-liquid system co-exist at equilibrium. Practically eutectic point shows the chemical composition and temperature corresponding to the lowest melting point of a mixture of components and is easily found in the phase diagram.
	
	Eutectic soldering has great properties for many use cases which makes it attractive solution for a variety of applications~\cite{handbook_of_wafer_bonding}. The main advantage of eutectic solders are their wetting properties, which allow for them to spread across a bonding area to fill the non-uniformities in the surfaces~\cite{handbook_of_wafer_bonding}. However, all oxides must be removed carefully to ensure good diffusion between the layers~\cite{franssila_introduction_2010}. A eutectic soldering process can be achieved by following a standard reflow process.~\cite{handbook_of_wafer_bonding}. This process works especially well for sensitive components because during bonding no physical forces are required to achieve a successful bond \cite{handbook_of_wafer_bonding}. Typical eutectic bonding process temperatures range between 200-600\celsius~\cite{handbook_of_wafer_bonding}
	
	
	
	
	
	
	\subsection{SLID bonding}
	    
	    Solid-liquid interdiffusion (SLID) bonding has received a lot of interest in the industrial and scientific community. SLID bonding has multiple desirable properties, such as\cite{handbook_of_wafer_bonding}:
	    
\begin{itemize}
\item high-temperature stability
\item moderate processing temperature
\item suitability for fine-pitch interconnects
\item thermodynamically stable bonds
\item corrosion-resistant bonds
\item suitability for wafer bonding
\item allows the usage of low-cost metallization
\item possibility for flux-free processes
\end{itemize}
	    
	    The SLID bonding process is based on a liquidous low temperature melting point metal mixing and solidifying with a higher temperature melting point metal. %The resultant intermetallic combination can be tested for stability and strength. Materials can be divided into low melting point metals and high melting point metals.
	    In a successful bond, all the low melting point material will have diffused into the high melting point material forming IMCs which increase melting point of the bond. \cite{handbook_of_wafer_bonding} Appendix~1 shows SLID process flow with the help of demonstrating figures.

        %The bonding process is performed in a temperature that the low melting point material is in liquid form and the high melting point material is solid. The materials diffuse into each other and form new intermetallics which have higher melting point and will solidify. The process has to be designed in a way that there are surplus of high melting point material and all the lower melting point material will diffuse and form IMCs.~\cite{handbook_of_wafer_bonding} 
        
        The SLID process shares some similarities with generic soldering processes including the existence of a liquid metal phase and IMC formation between the metals used. Typically, solders are based on tin, with small amounts of alloying materials such as copper. SLID can also utilize the same metal systems, but the proportions of the materials are opposite. %Even though SLID and soldering have similarities they are fundamentally different processes with unique properties. \cite{handbook_of_wafer_bonding}
        
        SLID bonding forms an IMC bond which melting point will significantly increase. The process becomes irreversible when diffusion of lower melting point material to the higher melting point material has created new IMCs and thus the bond is solidified. Suitable metal systems for SLID has certain solid solubility between the metals and it is typically in between 5 to 10~at\%. A stable temperature and compression are required for successful SLID bonds.The process parameters depend significantly on the selected metal system The most important parameters for this bonding method are temperature and duration, due to the fact that diffusion rate is a temperature driven phenomenon. SLID bonding supports limited self-alignment and 20~\micro m pitch has been demonstrated. SLID bonding has been demonstrated to be capable of hermetic bonds. Pad height and irregularity adaptability of a SLID bond is limited because of the small volume of a liquid metal phase during bonding. Reliability of the bond is dependent on the used metal system and formed IMCs. SLID bonds are thermo-dynamically stable and there are no changes in chemical compositions during thermal cycling, but IMCs are typically strong yet brittle due to their limited ductility. \cite{handbook_of_wafer_bonding}
        
        SLID bonding shares many target applications with anodic bonding and below table ~\ref{table:1} compares these technologies briefly.
\begin{comment}

    \begin{table}[h!]
    \centering
    \begin{tabular}{|c|p{5cm}|p{5cm}|}
    \hline
    Bonding method & Pros & Cons\\
    \hline
    Anodic Bonding & \begin{itemize}
    \item Well known process
    \item Reliable
    \item Hermetic bonds
    \end{itemize} & \begin{itemize}
    \item High voltage needed
    \item limited to specific glass materials
    \item CMOS incompatible
    \end{itemize}\\
    SLID Bonding& \begin{itemize}
    \item Low bonding temperature
    \item Suitable for fine pitch interconnection
    \item Strong bonds 
    \end{itemize} & \begin{itemize}
    \item Additional processing steps required
    \end{itemize}\\
    \hline
    \end{tabular}
    \caption{Anodic vs SLID bonding}
    \label{table:1}
    \end{table}

\end{comment}
	
\begin{table}[H]
\centering
\caption{Anodic vs SLID bonding}
\label{my-label}
    \begin{tabular}{|l|C{1.5cm}|C{1.5cm}|}
        \hline
        \textbf{Features}                 & \textbf{SLID} & \textbf{Anodic} \\ \hline
        Mature process                    &               & X               \\ \hline
        Reliable                          &               & X               \\ \hline
        Hermetic                          & X             & X               \\ \hline
        Suitable for Low temperature      & X             &                 \\ \hline
        Suitable for fine pitch           & X             &                 \\ \hline
        Strong bond strength              & X             &                 \\ \hline
        Suitable for all glass materials  & X             &                 \\ \hline
        CMOS compatible                   & X             &                 \\ \hline
        Requires no additional processing &               & X               \\ \hline
    \end{tabular}
\end{table}
\label{table:1}
\clearpage
		
\section{Study of Material Combinations} %% Mikael writes this

	%% Shortly about the material matrix collected, phase diagrams that acted as basis for selecting these and excluded materials (lead, mercury, chromium 6, cadmium).
	
	    Background research on different material combinations began as early as week 3 of the project. First steps were finding out if any existing lead-free solder materials could be used in SLID bonding. After that combinations based on copper, silver and gold as high temperature material were investigated. Finally a single combination using nickel as high temperature material was also considered. The research on these was heavily based on binary and ternary phase diagrams as well as articles about certain combinations used for SLID bonding already. The number of viable material combinations was limited by directive 2011/65/EU, which restricts the use of certain materials including lead, mercury, cadmium and hexavalent chromium.~\cite{noauthor_directive_2011} 
	
	    During materials research, all investigated combinations were listed in a material matrix, which included information about the bonding temperature, intermetallic compounds formed, advantages and disadvantages of said combination and sources for these. The material matrix is in Appendix~\ref{App:Matrix}. Material combinations in this matrix were discussed and most potential ones selected for further discussion with the instructor. The following subsections give details about each researched material combination.

	\subsection {Tin-Silver-Copper}

		Tin-Silver-Copper is commonly found in lead-free solder used in electronics manufacturing. Its melting temperature and other mechanical properties can be changed by varying compositions of the three elements. It does have a eutectic point at the Sn–3.81at\%Ag–1.66at\%Cu with melting point 217.7\celsius.~\cite{chen_phase_2006} This would mean that realistic bonding temperature would have to be at least 250\celsius, which is still not impossible, however lower temperature is preferred. If used for SLID bonding, copper would be the high temperature material and layers of tin and silver would be on top of a copper pillar. In such a case first interdiffusion would be between tin and silver, leading to \ce{Ag3Sn}, following a binary phase diagram in Figure~\ref{Ag_Sn_binary}. Then copper would diffuse into the mix, creating \ce{Cu6Sn5} and \ce{Cu3Sn}. Ternary IMCs between these elements are not known to exist~\cite{chen_phase_2006}.
		
	    \begin{figure}[H]
        \begin{center}
        \includegraphics[height=8cm]{Ag_Sn_binary.png}
        \end{center}
        \caption{Silver - Tin binary phase diagram~\cite{chen_phase_2006}}
        \label{Ag_Sn_binary}
        \end{figure}
        
        Advantages of Tin-Silver-Copper are its chemical stability and the fact that it is well known combination. Although it is commonly used for soldering and not SLID bonding. The disadvantages are difficulties in depositing silver, as Micronova does not have suitable machinery for this purpose and Also the bonding temperature would most likely be 250\celsius ~which is higher than some other combinations that were investigated.
		
		%% Phase Diagrams of Pb-Free Solders and their Related Materials Systems (Chen 2006).

	\subsection{Copper-Tin-Indium}
	
	    Copper-Tin-Indium is a natural next step from traditional Copper-Tin SLID bond. In this combination both tin and indium together are low temperature materials while copper is the high temperature material. As can be seen from Figure~\ref{In_Sn_binary} there exist temperatures even lower than melting point of indium in the mix of indium and tin. Especially between 42at\%Sn and 75at\%Sn where the mixture of \textbeta~ and \textgamma~ does not become liquid at all in pre-heat step under 120\celsius. When the actual bonding temperature is applied, this mixture becomes fully liquid under 200\celsius ~anywhere within specified range.
	    
	    \begin{figure}[H]
        \begin{center}
        \includegraphics[height=8cm]{In_Sn_binary.png}
        \end{center}
        \caption{Indium - Tin binary phase diagram~\cite{okamoto_desk_2010}}
        \label{In_Sn_binary}
        \end{figure}
        
	    While bonding, tin and indium would first melt into liquid and when copper dissolves into the mixture, \texteta-\ce{Cu6(SnIn)5}~\cite{liling_yan_hermetic_2008}. As seen from Figure~\ref{Sn_Cu_In_ternary}, this IMC is independent from the exact ratio between tin and indium as long as right amount of copper is present. This same mixture may also contain \ce{Cu2In}. IMC of \ce{Cu11In9} may have formed before \texteta~phase as well if the combination has much more indium than tin. \texteta-\ce{Cu6(SnIn)5} is stable in room temperature as is \ce{Cu11In9}~\cite{chen_phase_2006}\cite{liling_yan_hermetic_2008}\cite{bahari_equilibrium_2003}. According to Chen et al. \ce{Cu2In} is not stable at room temperature~\cite{chen_phase_2006}.

        \begin{figure}[H]
        \begin{center}
        \includegraphics[height=10cm]{Sn_Cu_In_ternary.png}
        \end{center}
        \caption{Tin - Copper - Indium isothermal section at 250\celsius~\cite{chen_phase_2006}}
        \label{Sn_Cu_In_ternary}
        \end{figure}
        
        Advantages of Copper-Tin-Indium are high solubility of tin and indium into each other and the low melting point caused by addition of indium. If the eutectic point of In-48.3at\%Sn is targeted while designing low temperature layer thicknesses, a successful bond at temperature 30\celsius ~higher than the eutectic temperature of 120\celsius ~should be achievable. Also all materials have electroplating baths available at the Micronova facility. Disadvantages are high oxidation of both indium and tin. This can be avoided by planning layer structures so that the most readily oxidizing indium does not come into contact with air. Liliang et al. solved this problem by adding 50~nm gold as topmost layer~\cite{liling_yan_hermetic_2008}.
	
	    %% Phase Diagrams of Pb-Free Solders and their Related Materials Systems (Chen 2006).
        %% The equilibrium phase diagram of the copper–indium system: a new investigation (Bahari 2002).
        %% Crystal Structure of Cu-Sn-In Alloys Around the g-Phase Field Studied by Neutron Diffraction (Aurelio 2012).
        %% A Hermetic Chip to Chip Bonding at Low Temperature with Cu/In/Sn/Cu Joint (Yan 2008).
        %% In-Sn phase diagram (Okamoto 2006).
        
	\subsection{Copper-Tin-Bismuth}
	
	    In this combination, tin and bismuth together form a low temperature alloy, while copper acts as the high temperature material. The basis for this combination is easily understandable from the binary phase diagram of tin and bismuth shown in Figure~\ref{Sn_Bi_binary}. The melting point decreases when the two are mixed, culminating in eutectic point at Sn-43at\%Bi which has melting point of 139\celsius. There are no intermetallics between tin and bismuth~\cite{chen_phase_2006}.
	    
	    \begin{figure}[H]
        \begin{center}
        \includegraphics[height=8cm]{Sn_Bi_binary.png}
        \end{center}
        \caption{Tin - Bismuth binary phase diagram~\cite{chen_phase_2006}}
        \label{Sn_Bi_binary}
        \end{figure}
        
        However, bismuth does not form IMCs with copper either, so in an actual bond there would be bismuth precipitation within the bonded area. All of the formed IMCs would be between copper and tin,  these being \ce{Cu6Sn5} and \ce{Cu3Sn}.~\cite{chen_phase_2006}
        
        The advantage of Copper-Tin-Bismuth is low melting temperature of Tin-Bismuth mixture. The disadvantage is the difficult deposition of bismuth. In addition, the only formable IMCs in this combination would be \ce{Cu6Sn5} and \ce{Cu3Sn}, which leaves bismuth precipitates in the bond.
        %% Need to find good sources for above claims about bismuth. Claims removed as incorrect.
	
	    %% Phase Diagrams of Pb-Free Solders and their Related Materials Systems (Chen 2006).
	   
	\subsection{Copper-Tin-Zinc}
	   
	   Copper-Tin-Zinc is another continuation of Copper-Tin SLID bonding. Adding zinc to tin reduces the melting point of the alloy to 198.5\celsius ~at eutectic point of Sn-14.8at\%Zn~\cite{chen_phase_2006}. Solubility of zinc into tin and vice versa are negligible. When copper is added to the mix, melting points below 230\celsius ~are maintained at some concentrations and binary IMCs form between copper and both zinc and tin.~\cite{chen_phase_2006} These are shown in Figure~\ref{Sn_Zn_Cu_ternary}. There is liquid phase at 230\celsius on some areas of the ternary phase diagram. There also exist Copper-Zinc IMCs \ce{CuZn}, \ce{Cu5Zn8} and \ce{CuZn5} as well as typical Copper-Tin IMCs \ce{Cu3Sn} and \ce{Cu6Sn5}. No ternary IMCs are  known to exist~\cite{chen_phase_2006}.
	   
	   \begin{figure}[H]
       \begin{center}
       \includegraphics[height=10cm]{Sn_Zn_Cu_ternary.png}
       \end{center}
       \caption{Tin - Zinc - Copper isothermal section at 230\celsius~\cite{chen_phase_2006}}
       \label{Sn_Zn_Cu_ternary}
       \end{figure}
       
       Some advantages of Copper-Tin-Zinc are low melting point of the eutectic composition between tin and zinc and the cheap price of zinc. The disadvantage is the rapid oxidation of zinc. Additionally, as tin does not mix with zinc to create IMCs, all tin has to be consumed by reactions with copper. If zinc reacts with all nearby copper first, it may be possible that pure tin is left in areas farthest from copper as diffusion through solid Cu-Zn IMCs is slower than dissolution through liquid.
	   
	   %% Phase Diagrams of Pb-Free Solders and their Related Materials Systems (Chen 2006).
	   
	\subsection{Silver-Indium}
	
	    Interest into Silver-Indium arose purely from their binary phase diagram, which is shown in Figure~\ref{Ag_In_binary}. Okamoto gives two different phase diagrams, his and M. R. Barren's from 1992 and D. Jendrzejczyk's from 2005~\cite{okamoto_desk_2010}. Both agree on general shape and liquidus curve, but Jendrzejczyk's version lacks \textbeta~ phase, \textxi'~ phase and \ce{Ag3In}. Okamoto admits that this has more accurate boundaries of \textxi~ phase~\cite{okamoto_desk_2010}. Based on this and being newer by 13~years, Jendrzejczyk's version was used for further investigation.
	    
	    
	    \begin{figure}[H]
        \begin{center}
        \includegraphics[height=8cm]{Ag_In_binary.png}
        \end{center}
        \caption{Silver - Indium binary phase diagram by Jenrzejczyk~\cite{okamoto_desk_2010}}
        \label{Ag_In_binary}
        \end{figure}
        
        Indium's melting point is enough by itself to allow bonding attempts under 200\celsius. If bonding is attempted at 200\celsius, \ce{AgIn2} does not form because it has melting point of 166\celsius. First IMC to form is \ce{Ag2In}, which melts at 312\celsius. Only the latter could survive reflow soldering process. If Barren's and Okamoto's 1992 work is correct and \ce{Ag3In} exists, it would have a melting point of 187\celsius, not being usable with reflow soldering.
        
        Advantages of this combination would be its simplicity and chemical stability of silver. Disadvantages would be high cost of silver, uncertainties about the binary phase diagram and inability to deposit thick layers at Micronova.
	    
	    %% Ag-In phase diagram (Okamoto 2006).
	   
	\subsection{Silver-Tin-Indium}
	
	    This combination is a logical continuation of Silver-Indium. By adding tin to indium, their melting point can be brought lower than either alone, as shown in Figure~\ref{In_Sn_binary}. In bonding process tin and indium would form \textbeta~ and \textgamma~, while some indium could react with silver to form mostly \ce{Ag2In}~\cite{chen_phase_2006}. As shown in Figure~\ref{Sn_Ag_In_ternary}, there is a continuous \textxi~ phase in the silver rich corner. This phase is an amalgamation of \textxi~phase of silver~\cite{chen_phase_2006} as seen in Figure~\ref{Ag_Sn_binary} and Figure~\ref{Ag_In_binary}. No ternary IMCs are known~\cite{chen_phase_2006}.
	    
	    \begin{figure}[H]
        \begin{center}
        \includegraphics[height=10cm]{Sn_Ag_In_ternary.png}
        \end{center}
        \caption{Tin - Silver - Indium isothermal section at 180\celsius~\cite{chen_phase_2006}}
        \label{Sn_Ag_In_ternary}
        \end{figure}
        
        Advantages of Silver-Tin-Indium would be its potentially low bonding temperature and silver's chemical stability. Disadvantages are high cost due to large amounts of silver if targeting the \textxi~ or \ce{Ag2In} and depositing those amounts.
	    
	    %% Ag-In phase diagram (Okamoto 2006).
	    %% Phase Diagrams of Pb-Free Solders and their Related Materials Systems (Chen 2006).
\clearpage

	\subsection{Gold-Indium-Bismuth}
	    
	    This combination uses bismuth and indium as a mix for the low temperature material. Gold is used as high temperature material. As can be seen from Figure~\ref{Bi_In_binary}, melting point decreases when adding indium to bismuth. Almost any combination with >50at\% indium could be used as low temperature material. Among these are \ce{BiIn2}, \ce{Bi3In5} or \ce{BiIn}.
	    
	    Aasmundtveit et al. have demonstrated Au-In-Bi SLID bonding in 90\celsius, 110\celsius and 120\celsius ~using In-Bi foil with eutectic composition Bi-78.5at\%In on Au pads. In their work IMCs formed were \ce{Au7In3}, \ce{AuIn} and \ce{AuIn2}. When bonding at 90\celsius, \ce{BiIn2} was formed, which melts at 89.5\celsius. If using higher bonding temperatures, bismuth stays as precipitates on its own.~\cite{aasmundtveit_-bi_2016}
	    
	    \begin{figure}[H]
        \begin{center}
        \includegraphics[height=8cm]{Bi_In_binary.png}
        \end{center}
        \caption{Bismuth - Indium binary phase diagram~\cite{aasmundtveit_-bi_2016}}
        \label{Bi_In_binary}
        \end{figure}
        
        Advantages of Gold-Indium-Bismuth are low bonding temperature, chemical stability of gold and >450\celsius ~melting point of Au-In IMCs~\cite{aasmundtveit_-bi_2016}. Disadvantages are the price of gold, as up to 3.5~µm may be needed for a successful bond and having to introduce In-Bi alloy as a foil. Also pure bismuth left in the bond center may threaten its mechanical reliability.
	    
	    %% In-Bi Low-Temperature SLID Bonding Knut E. Aasmundtveit* et al.
	   
	\subsection{Gold-Indium}
	    
	    Gold-Indium combination shows IMCs that are stable at reflow soldering temperatures, mainly \ce{AuIn2} and \ce{AuIn}~\cite{okamoto_desk_2010}. However, indium and gold don't have very good solubility to each other and the liquidus curve rises rapidly after the indium end as can be seen in Figure~\ref{Au_In_binary}. In SLID bonding this is not an issue as mix of pure indium and \ce{AuIn2} is partially melted above 156\celsius, so diffusion can take place. If bonding temperature is kept below 496\celsius, very few other IMCs should form besides \ce{AuIn2}.
	    
	    This is confirmed by Bernstein, who did bonding experiments between 200\celsius ~and 400\celsius, with intervals of 50\celsius~\cite{bernstein_semiconductor_1966}. Bonding at 200\celsius ~for 10~minutes resulted in major amounts of unreacted indium, while 30~minutes at the same temperature would have fewer of such, with actual \ce{AuIn2} present. With increases in temperature, more \ce{AuIn} and even \textgamma, \textxi~ and \textalpha~ phase were present.~\cite{bernstein_semiconductor_1966}
	    
	    \begin{figure}[H]
        \begin{center}
        \includegraphics[height=8cm]{Au_In_binary.png}
        \end{center}
        \caption{Gold - Indium binary phase diagram~\cite{okamoto_desk_2010}}
        \label{Au_In_binary}
        \end{figure}
	    
	    Advantages of this combination are sufficiently low temperature and chemical stability of gold. Disadvantages are price of gold and effect of potential unreacted materials weakening the bond strength. The latter can be countered by sufficiently long bonding time.
	    
	    %% Bernstein, Leonard. ”Semiconductor Joining by the Solid‐Liquid‐Interdiffusion (SLID) Process I . The Systems Ag‐In, Au‐In, and Cu‐In”. Journal of The Electrochemical Society 113, nro 12 (12. tammikuuta 1966): 1282–88. https://doi.org/10.1149/1.2423806.
	    %% Au-In phase diagram (Okamoto 2006).
	   
	\subsection{Gold-Tin}
	    
	    Gold-Tin is a tested material in hard solders, especially in optoelectronics applications~\cite{edward_j._wills_bonding_nodate}. When starting from the tin side of a Gold - Tin binary phase diagram, as shown in Figure~\ref{Au_Sn_binary}, first is \texteta ~phase, followed by \textepsilon~ phase and \textdelta~ phase. These are known as \ce{AuSn4}, \ce{AuSn2} and \ce{AuSn} respectively~\cite{yamada_formation_2004}. Problem with these is that they all start melting below 311\celsius \,and are as such incompatible with reflow soldering. Next is \textxi~ which melts at 522\celsius. This IMC requires 84at\%Au and thus long enough bonding time and high enough temperature. In experiment by Yamada et al. Tin-Gold-Tin diffusion couple provided \ce{AuSn4}, \ce{AuSn2} and \ce{AuSn} IMCs at 160\celsius~\cite{yamada_formation_2004}. Judging by the phase diagram in Figure~\ref{Au_Sn_binary}, \textxi~ would require over 190\celsius to form.
	    
	    \begin{figure}[H]
        \begin{center}
        \includegraphics[height=8cm]{Au_Sn_binary.png}
        \end{center}
        \caption{Gold - Tin binary phase diagram~\cite{okamoto_desk_2010}}
        \label{Au_Sn_binary}
        \end{figure}
        
        Advantages of Gold-Tin combination are its simplicity, chemical stability of gold and feasibly low bonding temperature to reach \textxi. Disadvantages are high amount of gold required, especially considering the price of gold. Another problem is depositing thick enough layers of gold for this to succeed.
	    
	    %% Yamada, T., K. Miura, M. Kajihara, N. Kurokawa, ja K. Sakamoto. ”Formation of Intermetallic Compound Layers in Sn/Au/Sn Diffusion Couple during Annealing at 433 K”. Journal of Materials Science 39, nro 7 (1. huhtikuuta 2004): 2327–34. https://doi.org/10.1023/B:JMSC.0000019993.32079.c2.
        %% Edward J Willis. ”Die Bonding for High-Power Devices | Solid State Technology”. Viitattu 5. maaliskuuta 2018. http://electroiq.com/blog/2002/05/die-bonding-for-high-power-devices/.
        
	   
	\subsection{Gold-Tin-Indium}
	    
	    This combination is a further development of Gold-Tin system, introducing indium lowers melting point of Tin-Indium alloy. This is evident by Figure~\ref{In_Sn_binary}. Isothermal section at 227\celsius shown in Figure~\ref{Sn_Au_In_ternary} gives evidence of liquid phase on Tin - Indium side. IMCs formed could be \ce{AuSn2}, \ce{AuSn}, \ce{AuIn2}, \ce{AuIn} and \ce{Au7In3}~\cite{chen_phase_2006}. Okamoto has named \ce{Au7Sn3} as \textgamma ~phase~\cite{okamoto_desk_2010}. A ternary IMC \ce{Au4In3Sn3} exists and is stable between 82 and 428\celsius~\cite{chen_phase_2006}. If this is not stable in room temperature, it makes a SLID bond potentially unreliable.
	    
	    \begin{figure}[H]
        \begin{center}
        \includegraphics[height=10cm]{Sn_Au_In_ternary.png}
        \end{center}
        \caption{Tin - Gold - Indium isothermal section at 227\celsius~\cite{chen_phase_2006}}
        \label{Sn_Au_In_ternary}
        \end{figure}
        
        Advantages of this combination are potentially low bonding temperature while getting stable IMCs \textxi ~(shown as hcp in Figure~\ref{Sn_Au_In_ternary}), \ce{AuIn2} and \ce{AuIn}. Of these, the Gold - Indium IMCs are preferable due to their higher melting point and less gold required. Disadvantages are price of gold, any unreacted pure tin and slow forming time of some Gold - Indium IMCs~\cite{bernstein_semiconductor_1966}.
	    
        %% this patent by Lichtenberger wasn't cited as I didn't notice anything useful in it. The compositions outlined there fall into AuSn + HCP triangle of ternary phase diagram and are not as such usable to SLID bonding.
        %% Lichtenberger, Heiner. Gold-tin-indium solder for processing compatibility with lead-free tin-based solder. United States US20110180929A1, filed 18. kesäkuuta 2009, ja issued 28. heinäkuuta 2011. https://patents.google.com/patent/US20110180929/en.
        %% Phase Diagrams of Pb-Free Solders and their Related Materials Systems (Chen 2006).
        %% Bernstein, Leonard. ”Semiconductor Joining by the Solid‐Liquid‐Interdiffusion (SLID) Process I . The Systems Ag‐In, Au‐In, and Cu‐In”. Journal of The Electrochemical Society 113, nro 12 (12. tammikuuta 1966): 1282–88. https://doi.org/10.1149/1.2423806.
	   
	   
	\subsection{Bismuth-Indium-Tin}
	    
	    Bismuth-Indium-Tin has originally been investigated for lead-free solder~\cite{effenberg_ternary_2006}. Due to this it ended up being considered for SLID bonding. As can be seen from isothermal section in Figure~\ref{Bi_In_Sn_ternary}, there exists liquid at least partially in many places at 76\celsius. This means that very low bonding temperatures could be achieved, but possibly all three metals could melt. As is already shown in Figure~\ref{Sn_Bi_binary}, tin and bismuth don't have any IMCs between them. Indium and bismuth have several IMCs between them, \ce{BiIn2}, \ce{Bi3In5} and \ce{BiIn}~\cite{aasmundtveit_-bi_2016}. None of these remain solid above 110\celsius ~as can be seen from Figure~\ref{Bi_In_binary}. No ternary IMCs are known to exist~\cite{effenberg_ternary_2006}.
	    
	    \begin{figure}[H]
        \begin{center}
        \includegraphics[height=10cm]{Bi_In_Sn_ternary.png}
        \end{center}
        \caption{Bismuth - Indium - Tin isothermal section at 76.45\celsius~\cite{effenberg_ternary_2006}}
        \label{Bi_In_Sn_ternary}
        \end{figure}
        
        Advantage of this combination very low bonding temperature. Potentially under 100\celsius. Disadvantages are lack of IMCs stable above 300\celsius. Bismuth also can remain as precipitate inside the bond, weakening it~\cite{effenberg_ternary_2006}. Also, bismuth is difficult to deposit and alloy with the other materials.
	    
        %% O Fabrichnaya, V Witusiewicz, J Gröbner. Bi-In-Sn (Bismuth-Indium-Tin). In: Effenberg G., Ilyenko S. (eds) Non-Ferrous Metal Systems. Part 3. Landolt-Börnstein - Group IV Physical Chemistry (Numerical Data and Functional Relationships in Science and Technology), vol 11C3. Springer, Berlin, Heidelberg
\clearpage
	\subsection{Gold-Antimony}
	    
	    Gold-Antimony combination makes use of eutectic point Au-35.5at\%Sb where liquid is present at 360\celsius~\cite{noauthor_development_nodate}~\cite{okamoto_desk_2010}. While this is lower than either melting point of gold or antimony, as seen in Figure~\ref{Au_Sb_binary}, it is not low enough for this project. Only known IMC is \ce{AuSb2}~\cite{noauthor_development_nodate}. This compound melts at 460\celsius, and would thus be usable.
	    
	    \begin{figure}[H]
        \begin{center}
        \includegraphics[height=8cm]{Au_Sb_binary.png}
        \end{center}
        \caption{Gold - Antimony binary phase diagram~\cite{okamoto_desk_2010}}
        \label{Au_Sb_binary}
        \end{figure}
        
        Advantages of Gold-Antimony are sufficient melting point of \ce{AuSb2} and chemical stability of gold. Disadvantages are too high bonding temperature and lack of methods for depositioning of thick layers of gold.
	    
        %% Paul, Aloke & Laurila, T & Vuorinen, Vesa & Divinski, Sergiy. (2014). Development of Interdiffusion Zone in Different Systems. 141-166. 10.1007/978-3-319-07461-0_4.
\clearpage
	\subsection{Nickel-Indium-Tin}
	    
	    Nickel-Indium-Tin is a continuation of Nickel-Tin SLID bonding, which has been used due to its cost effectiveness and compatibility with existing microfabrication methods~\cite{yoon_nickeltin_2013}. Successful bonds have been created at 250\celsius~\cite{yoon_nickeltin_2013}. Adding indium to the mix aims to lower bonding temperature. Any alloy with >20at\%In will melt at 200\celsius ~and can be used as the low temperature material. When that reacts with nickel, \ce{Ni3Sn4}, \ce{Ni2In3}, \ce{NiIn} and \ce{Ni3In} can be expected to form as interpreted from Figure~\ref{Sn_Ni_In_ternary}. Other IMCs like \ce{Ni28In72} and \ce{Ni13In9} have been reported~\cite{chen_phase_2006}. Okamoto recognizes \ce{In7Ni3}~\cite{okamoto_desk_2010} that can be deemed more accurate than \ce{Ni28In72}. Ternary compounds \ce{Ni6InSn5} and \ce{Ni3(In,Sn)4} have been found~\cite{chen_phase_2006}.
	    
	    \begin{figure}[H]
        \begin{center}
        \includegraphics[height=10cm]{Sn_Ni_In_ternary.png}
        \end{center}
        \caption{Tin - Nickel - Indium isothermal section at 240\celsius~\cite{chen_phase_2006}}
        \label{Sn_Ni_In_ternary}
        \end{figure}
	    
	    \ce{Ni3Sn4} melts at 794.5\celsius, \ce{Ni3Sn2} melts at 1160\celsius ~and \ce{Ni3Sn} melts at 920.5\celsius~\cite{yoon_nickeltin_2013}. All these are compatible with reflow soldering. \ce{In7Ni3} melts at 404\celsius~\cite{okamoto_desk_2010} and all other IMCs between indium and nickel have higher melting points than this.
	    
	    Advantages of this combination are price and deposition processes for nickel and tin, potentially low enough melting point and high melting points of IMCs formed. Disadvantages are complexity of the system and high oxidation rates of tin and indium.
	    
	    %% Phase Diagrams of Pb-Free Solders and their Related Materials Systems (Chen 2006). 
        %% Sn-In phase diagram (Okamoto 2006).
        %% In-Ni phase diagram (Okamoto 2006).
        %% Nickel–Tin Transient Liquid Phase Bonding Toward High-Temperature Operational Power Electronics in Electrified Vehicles (Yoon 2013)."
	       

\clearpage


\section {Analysis and selection of Material combinations} %% Mikael writes this
    
	%% Section to analyze our findings and select the combos for our experiments
	
	After background research on possible material combinations was completed, they were reviewed by the project team. Three best solutions were chosen to be presented to the instructor and advisor. After their comments and approvals, experimentation would proceed.
	
	The first choice was Copper-Tin-Indium. There were many reasons for this choice, mainly among them the good results of Liliang et al. and very low bonding temperature that could be achieved. As seen from Figure~\ref{In_Sn_binary} the eutectic point between tin and indium is in the middle, so getting reasonably close to that should be easy even by layer thickness standards of electroplating. Copper-Tin system is well tried and tested in Aalto, so there should not be many new parameters to change while trying to get Copper-Tin-Indium to work. Also, copper and tin electroplating baths already exist in Micronova, an there are plans to get an indium bath as well regardless of our decisions. As photomask on wafer limits deposition of these metals, no new etching chemical is needed for indium. Only possible seed layer of copper needs to be etched before bonding. When considering the IMCs this experiment would yield, they are not very complex and should be distinguishable with energy-dispersive X-ray spectroscopy (EDS). There are risks with oxidation of indium, but that could be countered either by limiting indium's exposure to air after electroplating, usage of thin gold layer to protect it or by applying more force to break oxide layers.
	
	The second choice was Nickel-Indium-Tin. This combination provides low enough bonding temperature as based on Figure~\ref{Sn_Ni_In_ternary} and experiences with Nickel-Tin bonding by Yoon et al. Success of a Nickel-Tin-Indium bond depends much on diffusion of tin and indium to nickel, creating IMCs that can withstand temperatures above 350\celsius. In practice this means \ce{Ni3Sn}, \ce{Ni2In3}, \ce{NiIn} and \ce{Ni3In}. Nickel is also cheap, easily to electroplate and the bath for electroplating it already exists in Micronova. While the binary IMCs are simple yet numerous, the two recognized ternary IMCs \ce{Ni6InSn5} and \ce{Ni3(In,Sn)4} add complexity. EDS analysis results of such compounds could be mistaken for different mixes of binary IMCs if care is not taken. For example \ce{Ni3(In,Sn)4} could be mistaken for \ce{Ni3Sn4} with unconsumed amounts of pure indium. From process standpoint Ni-In-Sn is expected to be straightforward.
	
	The third choice was Gold-Tin-Indium. While depositing the required layer thicknesses of gold might be problematic with equipment available at Micronova, the low temperatures possible make this interesting. Based on Figure~\ref{Sn_Au_In_ternary} suitable bonding temperatures could be 240\celsius~or 260\celsius~. The latter is based on Figure~\ref{Au_Sn_binary} and aims to skip \texteta~phase and utilize mix of \textepsilon~and liquid for increased diffusivity. From there IMC formation could proceed through \ce{AuIn} to \ce{Au7In3} or \ce{Au3In}. Even \ce{AuIn} is fully solid below 456\celsius~and if otherwise mechanically suitable, could be enough for a successful bond. Formation of Au-In IMCs further towards gold would take more time and be more unlikely to form in sufficient quantities unless bonding temperature and time are increased. Gold and tin are expected to be simple to process, indium could be more difficult to get as wanted. But all chosen material combinations include indium and thus include this same potential problem.
	
	


\clearpage

\section{Material Combination Experiments}
	
	Experiments were carried out to achieve a copper-tin-indium SLID bond based on the findings from the material research. After bonding samples were made to examine with both optical and scanning electron microscopes~(SEM). In preparation for the bonding experiments the project team completed Aalto Nanofab cleanroom and chemical training in addition to licensing for all of the required equipment for processing. Additionally, detailed process plans were created for each segment of the bonding experiment and twenty four 10~cm wafers were cleaned and provided, complete with a 300~nm \ce{SiO2} layer. This section will cover briefly the steps taken to achieve a low temperature SLID bond.
	
	\subsection{Wafer Processing}
	
	In order to achieve a successful bond, several steps were carried out and the wafers were inspected before and after each step. The first steps were backside lithography and a dry etching process to create a pattern on the backside for alignment and dicing of the wafers. Once these processes were completed and verified for quality, the frontside of the wafers was sputtered with a \ce{TiW} adhesion layer/diffusion barrier followed by a \ce{Cu} seed layer on top. Some wafers then underwent a frontside lithography process to prepare the wafer to be plated with features for bonding. Additionally some wafers skipped this process so that chip and wafer level bonds would be possible. After frontside lithography the wafers were electroplated with the high temperature material which is \ce{Cu}, followed by the low temperature materials \ce{In} on one side and \ce{Sn} on the other.  
	
	\subsection{Bonding and Microscopy}
	
	All bonding for this project was carried out at 250\celsius, though theoretically a bond should be achievable at 180\celsius~based on our research into the \ce{CuSnIn} system. Bonds were carried out at wafer and chip level and a reference sample of the system was reflowed just to look at the reactions. The low temperature material proportions and bonding times were altered to better understand the diffusion rates and stability of the system. After wafer level bonding was completed, the samples were diced, selected from several regions of the chip, and molded in epoxy for analysis. The molded samples were ground and polished to be examined under an optical microscope. Additional samples were sputtered with \ce{Cr} in preparation for SEM analysis. 
	
	Chip level bonds were carried at 100, 500, and 2500 seconds with 1:1 and 1:2 ratio of indium to tin. Chip level bondings allows for more precise control over the bonding parameters. This means that the wafers were diced before being bonded, allowing for better contact distance, more consistent pressure, and reduced bending, which would be seen at the wafer level. Chips were selected, molded in epoxy, ground,  polished, and \ce{Cr} was deposited for SEM analysis. Optical and SEM analyses were done on the samples to determine the quality of the bonds in the prepared samples. Lastly, EDS line and point scans were done with the SEM to characterize the chemical composition of the bonded samples.  
	
	
	
	
	
\iffalse
	
\section{To go to Appendix?}
	Before our project team could start processing the wafers we had to participate in two separate cleanroom trainings. Aalto Nanofab organized those trainings with the help of research groups. In addition to the trainings, almost every tool located at the Micronova cleanroom required license to operate. These licenses could be acquired from the tool main user by participating in a usage training. Trainings consumed significant portion of the total hours reserved for this project. However, there are several good reasons why this training is mandatory, such as:
	
	\begin{enumerate}
    \item To minimize the cleanroom and sample contamination risk
    \item To ensure proper usage of the machinery found in the cleanroom
    \item To offer guidance for all the required systems and software 
    \item To increase safety
    \end{enumerate}
	
	Before the actual processing started the team had created detailed process instructions which were followed during the processing. The process instructions consisted of the actual process instructions step by step, processing parameters and questions to be filled during the processing. These instructions were updated as our group learned more and changed plans to counter problems. Process instructions are found in appendix~2.
	
	The instructor of this project (Glenn Ross) prepared 24 wafers for our project. The wafers used were 10~cm in diameter. The wafers were cleaned with RCA and then thermally oxidized to form uniform 300~nm film of \ce{SiO2}.
	
	\subsection{Experiment 1}
	
	The first experiment consisted of six wafers which had been thermally oxidized. First experiment has been divided into major processing steps in the same order as it was processed.
	
	\fi
	
	\iffalse
    \subsubsection{Backside lithography}
    
    The process started with quick visual inspection to ensure that the wafers does not have anything unusual. The wafers were checked one by one and were placed on quartz rack and then the whole batch was primed in priming oven for 30~min to treat with HMDS to promote better adhesion of resist to the wafer. After priming the wafers were ready for resist coating and that was done with resist spinner using the following parameters:

    \begin{enumerate}
    \item Initial spinning rate 500~rpm
    \item duration 5~s
    \item Actual spinning rate 4000~rpm 
    \item duration 30~s
    \item AZ5214E photoresist
    \item roughly 3~ml photoresist was placed per wafer plus additional 1-2~ml in initial spinning phase
    \end{enumerate}
    
    All the wafers were inspected visually after the spinning. Considerable amount of resist cometing was present in every wafer and one of our wafers seemed unfit for the process, so it was cleaned with ultrasonic agitated acetone bath and isopropanol. After cleaning and drying the wafer was respinned and the quality was acceptable. Once all of the wafers were deemed satisfactory they were placed in 90\celsius ~oven for 20~minutes for a soft bake.  
    
    Once the wafers had been soft baked, mask aligning and exposure was done. Wafers did not yet have any patterns or alignment marks so the aligning was done roughly based on the wafer location compared to the mask. Exposure parameters were the following:
    
    \begin{enumerate}
    \item Exposure time 2.0~s
    \item Lamp power 350~W
    \item Plastic mask in a glass mask holder
    \item Soft contact
    \end{enumerate}

    After the exposure the wafers were transferred to a development bath. The bath contained AZ351B resist developer and wafers were gently moved inside the bath during the 60~seconds development. Once developing was complete, wafers were rinsed in cycling DI water container and dried with drying machine found at lithography finger. 
    
    Last step in the back side lithography was to hard bake all the wafers in 120\celsius ~oven for 30~min. After hard baking all the wafers were inspected visually to ensure that the patterns meet project quality requirements. %% Did they? 
    
    
    \subsubsection{Backside Etching}
    
    Our team chose to use RIE etching for the wafers. Our initial plan was to use HF wet etching because it can be done simultaneously for all of our wafers and it provides good selectivity against \ce{SiO2}. In the end we chose RIE because of the health risks of the HF. RIE machine found at Micronova cleanroom was capable of single wafer processing and we had to repeat the same process for all of our samples. In the beginning we did not know which recipe to use but then we found one meant to remove \ce{SiO2}. The recipe parameters were the following:
    
    \begin{enumerate}
    \item DC voltage started from 300~V but decreased to 200~V after couple of minutes of etching
    \item Etchant gas \ce{CF4} 
    \item Etching time was 15~minutes
    \item Processing pressure 30~mTorr
    \item Quartz plate was used
    \end{enumerate}
	
	This etching process removed the 300~nm layer of \ce{SiO2} selectively from unmasked areas. Ellipsometre was used to ensure the etching had proceeded as expected. 
	
	\subsubsection{Frontside sputtering and photoresist stripping}
	\label{exp1_frontside}
	
	After backside etching, our samples were sputtered with adhesion layer/diffusion barrier and with seed layer. There are other deposition methods that could have been used but sputtering is natural choice for the experiment because the film quality and deposition speed are optimal for this use case. In addition, sputtering can deposit more than one material in one cycle. Before the sputtering was started our team created a sputter recipe for the tool with the help of the main user of the tool. The following process parameters were used:
	
    \begin{enumerate}
    \item 20~nm TiW adhesion layer/diffusion barrier sputtered in 30~s
    \item 100~nm Cu seed layer sputtered in 1~min
    \item Sputter power target was set to 500~W
    \end{enumerate}
	
	After 5 wafers had been sputtered our team performed visual check and noticed mild patterns on the front side of the wafers. This subject is further discussed in discussion section. After this discovery, further investigation started and quite soon it was clear that the patterns are from the spinner plate. Either new batch must be processed or sputtered material has to be removed by etching followed by proper cleaning of the front side of the wafers. As the price of the processing materials and wafers were not a concern our team decided to start new batch from scratch. 

		
	\subsection{Experiment 2}
	
	
  \subsubsection{Backside lithography}
    
    The second batch started with backside lithography and extra care was taken into tools' cleanliness, especially the spinner plate. The second batch backside lithography process was almost identical to the first batch. The differences between the processes is listed below:
    
    \begin{enumerate}
    \item Spinner plate cleaned with acetone and isopropanol before use
    \item Before spinning wafers blowed with nitrogen gun to remove particles
    \item Slightly more photoresist poured to the wafer to decrease cometing
    \item Wafers inspected carefully after development
    \end{enumerate}

    Second batch of wafers had more uniform layer of photoresist compared to the first batch. Under the microscope there were some bubble shaped structures inside the photoresist which our team assumed to be from the plastic photomask. The size of the bubble shaped structures were small and they did not have any negative effect on our experiment. 

    
    
    
    \subsubsection{Backside Etching and photoresist stripping}
    
    Etching for the second batch was rather similar than for the first batch. However, major changes to the process parameters were done. The RIE recipe was changed into \ce{SF6} and the process parameters were the following:
    
    \begin{enumerate}
    \item Etchant gas \ce{SF6}
    \item duration 2~min
    \item DC voltage 200~V
    \item Power 100~W 
    \item Processing pressure 30~mTorr
    \item Graphene plate was used 
    \end{enumerate}
	
	This etching process removed the 300~nm layer of \ce{SiO2} and carved some silicon form the backside. Selectivity in this phase is not required because the backside pattern is used for aligning purpose only. During the RIE process one of the etched samples was inspected with optical microscope. Minor signs of plastic mask wear were spotted but nothing alarming was found. 
	 
	After RIE, wafers were prepared for resist stripping. In the second batch, order of sputtering and resist stripping phase was changed in order to use general stripping baths. Resist stripping was simple wet process and samples were basically moved from one bath to another after certain amount of time. Photomask stripping recipe was the following:
	
    \begin{enumerate}
    \item Ultrasound agitated acetone bath for 10~min 
    \item Cleaner acetone bath for 2~min moving jig gently around
    \item Isopropanol bath for 2~min moving jig gently around
    \item Dry wafers carefully.
    \end{enumerate}
	
	
	
	
	\subsubsection{Frontside sputtering and frontside lithography}
	
	Sputtering step was identical to the one performed for batch one (5\ref{exp1_frontside}). After sputtering the quality of the wafers was inspected and all six wafers fulfilled our quality requirements. 
	
	After the frontside sputtering, frontside lithography was performed. To have success in frontside lithography was significantly more important than it was for background lithography and all steps were performed extra carefully. In between every major step the quality was checked visually. The frontside lithography process followed the same process as background lithography with minor process parameter changes such as:
	
    \begin{enumerate}
    \item AZ5214E photoresist replaced with AZ15nXT
    \item Spinning speed changed from 4000~rpm to 2500~rpm, duration was not changed (30~s). Target photoresist thickness was 8~\micro m (8.5~\micro m measured with mechanical profilometer). Minor cometing in the corners of the wafers was visible which seemed to reduce during soft bake.
    \item Soft baking was done with hotplate at 110\celsius ~instead of oven. Duration decreased from 30~minutes to 3~minutes. 
    \item Hard baking was done with hotplate at 120\celsius ~for 1~minute.
    \item AZ351B developer replaced with AZ 726 MIF and developing was done three times for 55~seconds.
    \item Exposure time increased from 2~s to 20~s. 
    \end{enumerate}
	
	Frontside pattern was aligned regarding to backside pattern aligment marks. After the development no unexpected patterns or major visual defects were found. Once wafers passed visual inspection they were measured with mechanical profilometer from 5 different areas of the wafer. Table~\ref{table:1} below describes the the measurements.

    \begin{table}[h!]
    \centering
    \begin{tabular}{|c|r|r|r|r|r|}
    \hline
    Wafer number & Center & North & South & East & West\\
    \hline
    Wafer 1&X~\micro m&X~\micro m&X~\micro m&X~\micro m&X~\micro m\\
    Wafer 2&X~\micro m&X~\micro m&X~\micro m&X~\micro m&X~\micro m\\
    Wafer 3&X~\micro m&X~\micro m&X~\micro m&X~\micro m&X~\micro m\\
    Wafer 4&X~\micro m&X~\micro m&X~\micro m&X~\micro m&X~\micro m\\
    Wafer 5&X~\micro m&X~\micro m&X~\micro m&X~\micro m&X~\micro m\\
    \hline
    \end{tabular}
    \caption{Resist thicknesses}
    \label{table:1}
    \end{table}
	
\subsubsection{Electroplating}

Once thickness of the photoresist was measured, wafers were prepared to be electroplated. Sputtering leaves the surface of the wafer really smooth and shiny which causes risk of improper adhesion when electroplating. To avoid this issue all the wafers were cleaned and the surface roughened with oxygen plasma. Oxygen plasma process took couple of minutes per wafer with 100~W output power in low pressure oxygen environment. 

After plasma treatment, wafers were ready to be electroplated with copper. Formula~\ref{electroplating_time} shows how initial electroplating times were calculated. In this equation T~=~thickness, n~=~outer electrons, F~=~Faraday’s number (96485.3), \textrho~=~density, A~=~area, I~=~current and M~=~molar mass. Exposed area of our wafer was known and electroplating efficiency was assumerd to be 100\%.

\begin{equation}
t=(\dfrac{TnF\rho A}{IM10000})
\label{electroplating_time}
\end{equation}

Calculated values for copper, tin and indium are found below. 

Initial time for copper electroplating using Formula~\ref{electroplating_time}:

\begin{equation*}
t=\Bigg(\dfrac{5\times10^{-6}m\times2\times96485c/mol\times8,96\times10^{6}g/m^3\times300\times10^{-6}nF}{30\times10^{-3}A\times63,5g/mol}\Bigg)=22.7min
\end{equation*}

Initial time for tin electroplating using Formula~\ref{electroplating_time}:

\begin{equation*}
t=\Bigg(\dfrac{X \times10^{-6}m\times Y \times96485c/mol\times Z \times10^{6}g/m^3\times300\times10^{-6}nF}{30\times10^{-3}A\times W g/mol}\Bigg)=
\end{equation*}

Initial time for indium electroplating using Formula~\ref{electroplating_time}:

\begin{equation*}
t=\Bigg(\dfrac{X \times10^{-6}m\times Y \times96485c/mol\times Z \times10^{6}g/m^3\times300\times10^{-6}nF}{30\times10^{-3}A\times W g/mol}\Bigg)=
\end{equation*}

The first wafer was electroplated only for 18 minutes and 8 seconds to avoid excess copper on the pads. After the first electroplating process the layer thickness was measured with mechanical profilometer to get actual deposition rate. Based on the time and profilometer measurement results the copper deposition speed was 2.93~nm/s. Comparison to our calculations the rate is bit different which is probably caused because of the deposited are may differ from our initial assumption (SHOULD I ADD THIS TO RESULTS?). In addition optical profilometer was used to measure film thicknesses but it was abandoned because the measurement accuracy was not good enough and variance between measurements was high. The optical profilometer accuracy should in theory be good enough but the partly transparent photoresist may cause challenges to the measuring ???? bad sentence!. However, optical profilometer detected some signs of low copper additive and thus the container was stocked with new additives. In addition, signs of seedlayer corrosion was seen around the electrodes. This is caused by photoresist reaction with the elevated temperature and plating liquids. This challenge was solved with one extra step in which photoresist is stripped around the edges. 

The following samples were electroplated with copper using these process parameters,
\begin{enumerate}
\item Current set to 21.1~mA 
\item Open circuit voltage 5~V
\item Operating voltage 0.3~V
\item Flow adjusted to be laminar and uniform in all directions
\item Electroplating duration 22~minutes 45~seconds
\item Strip photoresist around the edges with NI7xx
\end{enumerate}

After all six samples were plated with copper, layer thicknesses were measured with the mechanical profilometer and the results are shown below in Table~\ref{table:2}. 

\begin{table}[h!]
\centering
\begin{tabular}{|c|r|r|r|r|r|}
\hline
Wafer number & Center & North & South & East & West\\
\hline
Wafer 1&X\micro m&X\micro&X\micro m&X\micro m&X\micro m\\
Wafer 2&X\micro m&X\micro&X\micro m&X\micro m&X\micro m\\
Wafer 3&X\micro m&X\micro&X\micro m&X\micro m&X\micro m\\
Wafer 4&X\micro m&X\micro&X\micro m&X\micro m&X\micro m\\
Wafer 5&X\micro m&X\micro&X\micro m&X\micro m&X\micro m\\
\hline
\end{tabular}
\caption{Copper layer thickness table}
\label{table:2}
\end{table}

After copper electroplating, setting up indium bath was required. Indium electroplating bath was not as robust compared to the copper and tin electroplating baths found at Micronova. Initializing consisted following steps:

\begin{enumerate}
\item Filling the tanks
\begin{enumerate}
\item Big tank with water
\item Smaller tank with indium containing acidic liquid
\item another small tank with DI water
\end{enumerate}
\item Initializing mixer mechanisms and blades
\item Place indium pellets to the holder
\item Setup and test current source
\item Place sample wafer into the wafer holder
\end{enumerate}

First indium electroplating attempt was done with these parameters:

\begin{enumerate}
\item Deposition time 61~seconds. This is based on previous test with the bath. Our calculated values is just to give glimpse about the magnitude.
\item Mixer speed 700~RPM
\item Current 63.3~mA
\item Liquid temperature 60\celsius 
\end{enumerate}

The first indium electroplating test was disappointment. The process destroyed the photoresist around the bonding bumps and indium was deposited around the bumps and photoresist cracks. After investigating this issue, processing continued with major changes in operating parameters. Specific operating parameters are found from the table X below,

\begin{table}[h!]
\centering
\begin{tabular}{|c|r|r|r|r|}
\hline
Wafer number & Mixer speed & Current & Duration & Liquid temperature\\
\hline
Wafer 1&500~RPM&62.5~mA&1min~1s &60\celsius \\
Wafer 2&1000~RPM&20.5~mA&3min~3s&60\celsius \\
Wafer 3&1000~RPM&6.5~mA&9m~9s&60\celsius \\ 
\hline
\end{tabular}
\caption{Indium electroplating process parameteres}
\label{table:3}
\end{table}

None of these tests succeeded in a way that wafers could be plated with tin or bonded. The last test even contaminated the bath with photoresist strips and filtering had to be done. However, with the help of the electroplating bath supplier and these test results the cause was identified. The photoresist did not withstand the plating solution. The photoresist used on our samples was developed for electroplating and was supposed to endure the electroplating solution. The supplier recommended to try overexposure with 3 times 20~s overexposure and then plating with hotplate for X~min. However, even with overexposure and extended hot plate time, electroplating did not fully succeed. 

To proceed with the experiment, minor changes to the process flow had to be made. Originally tin was meant to be electroplated on top of indium to minimize indium oxidizing, but because the photoresist did not stay on during the indium deposition, indium and tin were deposited on completely different wafers. Having tin and indium on different wafers should not affect bonding success significantly. 

Tin electroplating process followed the same process as copper electroplating and only electroplating time was different. Electroplated copper surface had enough texture that tin did adhere to that well enough and no oxide plasma treatment was required. 

After two wafer pairs had been successfully electroplated, the sputtered seed layers were etched away with wet etching. (Every paragraph should have at least two sentences so this one should have one more)

\subsubsection{Bonding and dicing}
 
First bonding attempt did not succeed as adhesion between bonds was poor. Second bonding attempt was done with minor changes to the process parameters and a somewhat successful bond was formed. Alignment accuracy was good and no major sliding occurred during the bonding. Bonding parameters were changed in order to ensure good diffusion between layers while still being low temperature process ($\leq$250\celsius). Process parameters:

\begin{enumerate}
\item Pre-heat to 100\celsius
\item Align wafers using two IR cameras
\item Bonding temperature 250\celsius
\item Ramp up 30\celsius ~per minute
\item Contact force 5~kN 
\item Bonding duration 60~min
\item Chamber cooled with nitrogen
\end{enumerate}

Adhesion between the wafers was good enough that dicing could be done. Backside lithography had created dicing lines which made dicing process easier. Dicing saw operated with manufacturer recommended operating parameters and dicing process was successful. Yield rate for successfully bonded chips was approximately 40\% . Our wafers are divided into north, east, south, west and center zones. South and east zones had highest amount of good chips. Three sample chips were molded into epoxy.

\subsubsection{Grinding and polishing}

After the chosen samples were moulded and dried, grinding was started. Grinding is simple process in which molded samples surface are refined with different roughness sandpapers. The process naturally starts from rough sand paper and which is used to grind all the excess material before the interesting part of the bond is seen. To smoothen the marks of rough sandpaper, paper with bit less roughness is needed. This same process is continued as long as there are only significant marks from the smoothest paper. This process includes manual work and must be done carefully to ensure that no failures to the process will happen. Typical challenges in grinding process are double surfaces or overgrinding through the chip. Project grinding process performed by using sandpapers from 320 to 4000 roughness and and after each step the results were checked with microscope before moving to next sand paper. 

After only marks from the smoothest papers were seen and no double surfaces was found, polishing was started. Polishing process is similar than grinding but instead of sandpaper, diamond particles containing liquids interacts with the surface and smoothens it. In addition lubricants are required for the process to perform smoothly. Setting up polishing machine and performing the process was simple. Our polishing consisted of four steps which each of them were bit smoother than the previous one (6~\micro m, 3~\micro m, 1~\micro m and 0.25~\micro m). With 10~min step time no issues detected and the process resulted in smooth and shiny surfaces. 

\subsubsection{Microscope and SEM pictures}

After polishing have been done the samples were examined with optical microscope and later with SEM. First, all the samples were inspected with optical microscope and the most interesting samples were chosen. Then these samples were coated with chromium to be able to inspect with SEM. 

SEM was used for examine bonds with higher magnification compared to optical microscope. In addition Energy-dispersive spectroscopy (EDS) was used to distinguish element compositions form points and along the specified lines. 


- Thickness  values from cleanroom computer
- Bonding third time parameters
- dicing
- 


    \subsection{Experiment 3}

Third batch for the experiment had to be made because almost every wafer had been used for indium electroplating without success. In addition, required tools were available only for a brief moment which had to be used. Third batch processing followed exactly same steps than batch two. All the processing steps were successful with batch 3 and its quality was great. However this batch was not patterned from the front side and and it was electroplated without patterns. Bonding was done on chip level and separate furnace was used to form the bond. Another five chip level bonds were chosen of these and they were grinded and polished for closer look in optical microscope and SEM. The process steps were almost identical to experiment 2.  




(However, batch 3 was not required for bonding as partially successful indium electroplating was done and deeper research behind the issues must have been made but withing this project scope this was not possible. Ensure this from Glenn!)





\clearpage

\fi

\section{Experimentation Results}

The project team was able to successfully produce SLID bonded samples using the researched metal system(\ce{CuSnIn} at a temperature of 250\celsius). This section discusses the experiment results in some detail.  

\iffalse
\subsection{Measurement methods}

Multiple different measurements methods were used in order to inspect our samples in detail. The primary tool to inspect results during processing was optical microscope. There were couple of different optical microscopes available in Micronova and most of them had fairly good magnification and optics to produce sharp images. Optical microscopes were able to provide images in the range of 50x to 1000x magnification. Figure~\ref{50x example} and Figure~\ref{1000x example} represent minimum and maximum magnifications. 

\begin{figure}[H]
        \begin{center}
        \includegraphics[height=8cm]{measurementmethods1.jpg}
        \end{center}
        \caption{Sample at 50x magnification}
        \label{50x example}
\end{figure}

\begin{figure}[H]
        \begin{center}
        \includegraphics[height=8cm]{measurementmethods2.jpg}
        \end{center}
        \caption{Sample at 1000x magnification}
        \label{1000x example}
\end{figure}

From the Figure~\ref{1000x example} narrow depth of focus is clearly seen, which is an unavoidable phenomenon when high magnification is used. However, this does not restrict our inspection as focus can be changed to match inner surfaces as well. The narrow depth of focus is an issue of optical microscopes in general for higher magnifications.

In addition to optical microscope, SEM was operated to obtain higher magnification and EDS material analysis. Indium and tin are close to each other in the periodic table and differentiating between them with SEM is difficult. However, EDS material analysis did identify presence of both materials. SEM was operated with 500x to 8000x magnification and Figure~\ref{SEM example} represents typical obtained SEM picture. 

\begin{figure}[H]
        \begin{center}
        \includegraphics[height=8cm]{SEMexample.jpg}
        \end{center}
        \caption{SEM example}
        \label{SEM example}
\end{figure}

During the processing multiple tools were used for inspecting process step success. For example profilometers were used to probe layer thicknesses. 


\subsection{Copper electroplating results}

Figure~\ref{Copper bump} presents the first copper electroplating results observed with optical microscope. Unintentional structure is seen in the middle that resembled a bump. These structures were inspected with greater detail using optical profilometer and as a conclusion, these bumps do not significantly affect our bonding performance. Majority of bonding pads had those structures visible and copper surface seemed to be bit rougher than typically. Increasing additive to the copper electroplating solution solved this issue and quality of the following electroplated wafers was satisfactory.

\begin{figure}[H]
        \begin{center}
        \includegraphics[height=8cm]{Weirdbumb.jpg}
        \end{center}
        \caption{Anomalous bump after copper electroplating}
        \label{Copper bump}
\end{figure}


\subsection{Indium electroplating results}

The largest challenges that our project team faced during experiments were with indium electroplating. First indium electroplating results are shown in Figure~\ref{Indium plating result} and appendix x. 

\begin{figure}[H]
        \begin{center}
        \includegraphics[height=8cm]{Indiumplating1.jpg}
        \end{center}
        \caption{Unusual copper formation in the middle of the bump}
        \label{Indium plating result}
\end{figure}

The bonding pad and resist next to it seems burned and there is lots of photoresist cracking around it. Clearly the electroplating bath solution attacks the photoresist chemically. Photoresist utilized in the experiment (AZ15nXT) is designed for electroplating and it was supposed to be strong enough for all typical plating solutions available. Further investigating and testing was done to find solutions for this challenge, and major process parameter changes were made, but no complete solution was found. However, in the end with changed processing parameters four wafers were successfully electroplated with indium. The uniformity and layer thickness were not good compared to tin and copper equivalents.
\fi

\subsection{Wafer level bonding results}

Bonding at wafer level was not entirely successful, as it resulted in rather low 40\% yield among the sample chips selected from different regions of the diced and bonded wafer. (northeast, northwest and southeast). Low yield rate is explained with poor bond quality which is considered to be caused from shortage of indium in the bond.

First sample was from northeast region (NE) of the wafer and Figure~\ref{NEWB1} represents typical of the bond quality achieved from the wafer level experiment. 

\begin{figure}[H]
        \begin{center}
        \includegraphics[height=10cm]{NEwaferbond1.jpg}
        \end{center}
        \caption{Optical microscope figure of wafer level Bond from sample NE region}
        \label{NEWB1}
\end{figure}

Inconsistencies were observed between different locations in the bonds. The copper layer in the top wafer in Figure~\ref{NEWB1} appears to be rougher than the bottom wafer, perhaps due to inconsistencies in the copper plating bath. %Another feature of the bond was that was observed was a grey bar located at the upper diffusion barrier (the copper and silicon interface at the figure). There was no unambiguous explanation for this bar, though a diffusion barrier failure was theorized. However, following observation with SEM revealed that diffusion barrier had remained intact. EDS revealed that the composition of this bar is copper and is possibly just an unusual structure of copper or a partially separated copper layer exacerbated by the lack of a depth of focus on the microscope. 
Deeper observation revealed, that only minor changes are seen across the samples. Figure~\ref{SEWB3} added as comparison to Figure~\ref{NEWB1}%However, this aforementioned grey bar is visible in only samples from north side and as a comparison Figure~\ref{SEWB3} is from southeast sample.

\begin{figure}[H]
        \begin{center}
        \includegraphics[height=10cm]{SEwaferbond3.jpg}
        \end{center}
        \caption{Optical microscope figure of wafer level Bond from sample SE region}
        \label{SEWB3}
\end{figure}

SEM analysis revealed additional information about the state of the bonds. The indium and copper surface area showed nearly complete voiding in the interfacial structure and contact of the bond was limited. The main reason for this feature was considered to be lack of indium in the system. Insufficient amount of indium did not stabilize the bond enough to create bond between the high and low melting point materials. This issue no longer occurred in later experiment where the system had bit thicker indium layer. %The Indium and copper surface area showed nearly complete voiding in the interfacial structure and contact of the bond was limited. One theory regarding the cause of this was that the bonding pressure maamount of indium indium layer was quite thin and force required to break indium's native oxide layer may have been too high. This is not an unambiguous explanation and other reasons are possible as well. 
This phenomena is shown in two SEM images, Figure~\ref{NEWBSEM1} and Figure~\ref{NEWBSEM2}.

\begin{figure}[H]
        \begin{center}
        \includegraphics[height=8cm]{NESEM1.jpg}
        \end{center}
        \caption{SEM figure of wafer level Bond from sample NE region}
        \label{NEWBSEM1}
\end{figure}


\begin{figure}[H]
        \begin{center}
        \includegraphics[height=8cm]{NWSEM1.jpg}
        \end{center}
        \caption{SEM figure of wafer level bond from sample NW region}
        \label{NEWBSEM2}
\end{figure}

In addition to SEM images, EDS analysis was done to better determine the distribution of materials within the bond. The EDS analysis results were close to what was expected except that indium layer thickness had been maybe one tenth of the target thickness. In most of the areas indium seemed to have dissolved into tin, forming tin \textbeta ~crystal structure areas. During the analysis single a single indium rich area was found with 30at\% indium. This area most likely had a different crystal structure because maximum solid solubility of indium into tin \textbeta ~crystal structure is less than 10at\%. However, the amount of indium found from the bonds was less than what was expected and EDS analysis results refers to non-uniform indium layer. The analysis did not exclude the possibility that indium oxide had been the reason for improper contact formation, as a slight increase in oxide concentration was observed closer to indium side of the bond. Figure~\ref{LINE1} represents line scan of the bond and Figure~\ref{LINE2} is the same line scan added to SEM image with scale.

\begin{figure}[H]
        \begin{center}
        \includegraphics[height=8cm]{LINESCAN1.jpg}
        \end{center}
        \caption{Line scan of the wafer level bond sample NW region}
        \label{LINE1}
\end{figure}


\begin{figure}[H]
        \begin{center}
        \includegraphics[height=8cm]{LINESCAN1andSEM.jpg}
        \end{center}
        \caption{Line scan and SEM imageof the wafer level bond sample NW region}
        \label{LINE2}
\end{figure}

Addition figures, point scan, and area measurement information is available in appendix~\ref{App:SEM}. 


\subsection{Chip level bonding results}

After concluding that wafer level bonding results were not what was desired,  chip level bonding using the same copper-indium-tin material combination was attempted and analyzed. Chip level bonding provided flexibility for process parameter design. %In chip level bonding pattering is not required and thus photoresist issue is avoided. 
Though Chip level bonding is not suitable for any practical application, it is capable of  proving the concept of this material combination. In chip level bonding, samples were divided to different bonding times and different tin layer thicknesses. Five chip level samples were inspected in detail.

Bonding force was applied through use of a spring loaded rig and as such was not varied between the chip level samples. Using a large copper weight for compression would result in an increased thermal mass, which would change the other bonding parameters significantly. Due to the rig, bonding force was not uniform across the entire surface of the chip and some of the samples had signs of liquid metal flow within the bond structure.
%%Chip level samples indicated that bonding force was not sufficient for creating uniform bonds.
Areas where both surfaces had contact, were distributed mostly on the edges of the samples. In the chip level samples there were significant differences on the bond quality and this subsection mostly focuses on bonded samples because they had the best bond quality out of all our samples.
%and less than 10\% of the bonding area had contact. 
%Chip level bonds had significantly less voids compared to wafer level bonds. However, the lack of contact area makes these bonds weak. 
The quality of the chip level bonds achieved, exceeded expectations. Samples were divided based on bonding time bonding time and tin and indium ratio. Following figure~\ref{CHIPSEM1}, figure~\ref{CHIPSEM3} and figure~\ref{CHIPSEM2} illustrate the  differences in the results observed between the samples with a 1:2 ratio of \ce{In} to \ce{Sn} at 100s, 500s, and 2500s. %Figure~\ref{CHIPSEM1} and Figure~\ref{CHIPSEM2} are from the same sample but from different locations. 

\begin{figure}[H]
        \begin{center}
        \includegraphics[height=8cm]{ChiplevelSEM1.jpg}
        \end{center}
        \caption{SEM figure of a chip level bond. Sample was plated with 1:2 ratio of indium and tin. Bonding time was 100s with compressive force}
        \label{CHIPSEM1}
\end{figure}

\begin{figure}[H]
        \begin{center}
        \includegraphics[height=8cm]{ChiplevelSEM3.jpg}
        \end{center}
        \caption{SEM figure of a chip level bond. Sample was plated with 1:2 ratio of indium and tin. Bonding time was 500s with compressive force}
        \label{CHIPSEM3}
\end{figure}

\begin{figure}[H]
        \begin{center}
        \includegraphics[height=8cm]{CHIPSEM22.jpg}
        \end{center}
        \caption{SEM figure of a chip level bond. Sample was plated with 1:2 ratio of indium and tin. Bonding time was 2500s with compressive force }
        \label{CHIPSEM2}
\end{figure}

Figure~\ref{CHIPSEM1} clearly shows improper contact area in the middle. However 100s is very short time for bonding and it is expected that bonds were not in liquid form during bonding. Figure~\ref{CHIPSEM3} shows that after 500s bonds have definitely improved and different phases can be seen easily. The bond with 2500s bonding time is shown in Figure~\ref{CHIPSEM2} and very promising results are seen. Diffusion between materials have been sufficient and different phases are in horizontal lines. However, some minor voiding is visible and completely expected.
%Figure~\ref{CHIPSEM1} clearly shows improper contact area in the middle.As comparison, contact area in Figure~\ref{CHIPSEM2} is significantly better. Improper contact was seen in wafer level bonding also, but the location was closer to the copper surface. Microscope pictures indicate that indium layer thickness observed in the wafer level bonds was significantly thinner than that of chip level bonds. Overly thin indium layer could cause improper bonding contact for two different reasons. First, too small indium concentration in the liquid solution would decrease tin and copper diffusion and then reduce IMC formation rate. Second, indium native oxide layer is fragile and usually broken during the bonding process. However, if the indium layer is thin, compressive bonding force may not be enough to break it because indium itself deforms under the force. Patterning indicitive of different phases within the bond structure can be observed in Figure~\ref{CHIPSEM3}.

%\begin{figure}[H]
%        \begin{center}
%        \includegraphics[height=8cm]{ChiplevelSEM3.jpg}
%        \end{center}
%        \caption{SEM figure of chip level bond with interesting patterns}
%        \label{CHIPSEM3}
%\end{figure}


Further material analysis was done and compositions from the points P8-P12 figure~\ref{CHIPSEM3},~\ref{CHIPSEM2} were analyzed with EDS. Figure~\ref{Spectral1} and Figure~\ref{Spectral2} represent point P11 and P12 EDS analysis spectral outcome of this scalloped bond structure.

%The two separate phases were clearly distinguishable in both SEM and optical microscope images. Typically separate phases in SLID bonding are located in horizontal lines but in this case the phases are separated by scalloped pattern. Figure~\ref{Spectral1} and Figure~\ref{Spectral2} represent point P11 and P12 EDS analysis spectral outcome of this scalloped bond structure. 


\begin{figure}[H]
        \begin{center}
        \includegraphics[width=15cm]{Spectrum_P13.jpg}
        \end{center}
        \caption{Spectral analysis of P11. Indium and tin are on top of each other and indium as a minority is not seen.}
        \label{Spectral1}
\end{figure}

\begin{figure}[H]
        \begin{center}
        \includegraphics[width=15cm]{Spectrum_P12.jpg}
        \end{center}
        \caption{Spectral analysis of P12. Indium and tin are on top of each other and indium as a minority is not seen.}
        \label{Spectral2}
\end{figure}

Spectral analysis reveals that the most significant difference between the two phases is the amount of copper in them. Phase found in P11 contains 55at\% of copper and phase in P12 contains less than of 1at\% copper. Phase P11 is expected to be \ce{Cu_6Sn_5} IMC and phase P12 is basically just tin. Indium content on both phases is similar at roughly (5at\%). Figure~\ref{CHIPSEM3} points out that 500s is not enough time to form a reliable bond.

Phases in the figure~\ref{CHIPSEM2} are divided horizontally and they seem significantly more stable compared to phases in figure~\ref{CHIPSEM3}. Indium content seems constant across the bond excluding the copper pads. According to EDS analysis, phase found at P9 is most likely \ce{Cu_6Sn_5} and phase around P8 and P10 is \ce{Cu_3Sn}. No significant amount of tin or indium residues are seen in figure~\ref{CHIPSEM2} and even shorter bonding time should be sufficient to form reliable bond.

So far we have been analyzing bond with 1:2 ratio of tin and indium. Bond with 1:1 ratio of indium were also considered as successful and actually the quality of it was almost identical to the bond with 1:2 ratio.

\begin{figure}[H]
        \begin{center}
        \includegraphics[width=12cm]{BondTin1Indium1_500s.jpg}
        \end{center}
        \caption{SEM figure of a chip level bond. Sample was plated with 1:1 ratio of indium and tin. Bonding time was 500s with compressive force.}
        \label{SEM_In1_Sn1_500}
\end{figure}

\begin{figure}[H]
        \begin{center}
        \includegraphics[width=12cm]{BondTin1Indium1_2500s.jpg}
        \end{center}
        \caption{SEM figure of a chip level bond. Sample was plated with 1:1 ratio of indium and tin. Bonding time was 2500s with compressive force.}
        \label{SEM_In1_Sn1_2500}
\end{figure}

With the exception of the tin layer thickness, the bonds shown in figures ~\ref{SEM_In1_Sn1_500} and ~\ref{SEM_In1_Sn1_2500} share identical bonding parameters with figures~\ref{CHIPSEM3} and ~\ref{CHIPSEM2}. Even the distinguishable phases consists of similar compositions and the most significant difference between samples are that chips with 1:1 ratio of tin and indium has slightly higher concentration of indium present, as expected. Indium concentration around tin rich areas are close to 7 at\% and copper rich areas close to 5 at\%, which is significantly less than what was expected.

According to the EDS analyses, chip level bonding low melting point compound had more indium compared to wafer bonding equivalent. However, indium percentage was much lower than expected for both experiments and the bonds consisted mostly of copper and tin. Despite having improper bonding force and an thinner indium layer than hoped, these bonding results were extremely promising. The line scan results and more SEM figures can be found in appendix~\ref{App:SEM}.% and practically all indium was dissolved into tin and the low melting point compound stayed in tin \textbeta ~crystal structure. There was diffusion between low melting point and high melting point metal systems but it wasn't enough \textcolor{red}{to} create reliable SLID bond. 

%EDS line scan results for chip level bonding indicates better diffusivity between materials compared to wafer level bonding. This could be explained with thicker low melting point layer. Line scan results and more SEM figures can be found in appendix x. 


\clearpage


\section{Discussion and Reflection}
	
	\subsection{Discussion}
	
	The initial scope of the project was to design a low temperature SLID bonding process complete with methods and parameters. However challenges and time constraints have forced the scope to be adjusted to analyzing a chip or wafer level bond from a material perspective. This section will discuss the challenges faced, solutions considered, and future considerations for this type of bonding solution.
	
	The first challenge was that project team however was fairly inexperienced even in established processes for this type of wafer processing. The team members took part in cleanroom and chemical training for permission to access and use the cleanroom in Aalto's Micronova facilities. Once the cleanroom training was complete, the team determined which machinery would be required for the processing work and applied for licenses and training for them. Information regarding the processing steps and specifics is discussed briefly in section 5: Material Combination Experiments. 
	
	Something that was discovered is that even if process plans are explicitly followed, issues will still arise. The first of the processing challenges that were faced, occurred during the very first backside lithography session. This was the first time that the team members had attempted this type of processing. Therefore some cometing was visible on the backside of the wafers after processing, indicating a non-uniform photo-resist mask. This issue was not an issue for the backside of the wafer, however it is an issue when doing front side lithography. The second and more serious challenge faced during this phase, was a visible pattern on the wafer's front side by the naked eye and appeared to match the pattern on the photo resist spinning plate. The team did further investigation to determine the cause of the contamination.The pattern can be observed in Figure~\ref{Polymer_Microscope}.  

    \begin{figure}[htb]
        \begin{center}
        \includegraphics[width=12.5cm]{Poly_Micro.jpg}
        \end{center}
        \caption{Microscope images of suspected Polymer contamination}
        \label{Polymer_Microscope}
    \end{figure}
    
    Once the contamination was inspected with a microscope it became evident that the contamination occurred because of improper cleaning of the spinning plate by the previous user. Individual droplets are visible in the pattern of the spinning plate indicating that there were remnants remaining from the previous user's process. This user was spinning a transparent polymer that was not immediately visible after the processing step. For future uses of the photo-resist spinner the team had agreed to do an additional cleaning beforehand to prevent such setbacks in the future. 
    
    Once new wafers were processed and sputtered with \ce{TiW} and \ce{Cu}, front side lithography was completed and they were electroplated with \ce{Cu} to form the high temperature material pillars. Once this process was complete the team was tasked with plating the pillars with a 0.9~\textmu m layer of \ce{In} with an untested indium electroplating bath. The first effort at plating \ce{In} was not successful despite following the manufacturer's instructions for the indium bath. The \ce{In} was plated, however it was visible to the naked eye that the pillars were misshapen. The sample was looked at under a microscope and it appeared that the resist mask was damaged during plating. It was hypothesized that this damage could have occurred due to too high flow rate in the indium bath. Adjustments were made to the flow rate and another wafer was plated.
    
    
    Reducing the flow seemed to have a negative effect on the results. When looked at under some magnification in Figure\ref{Indium plating result}, it appeared that the plating process with a lower flow rate resulted in charring of the mask around the pillars. This caused the mask to deform and \ce{In} was plated in areas of the wafer other than the pillars.
    
     \begin{figure}[H]
        \begin{center}
        \includegraphics[height=8cm]{Indiumplating1.jpg}
        \end{center}
        \caption{Damage to the mask as a result of the indium plating bath}
        \label{Indium plating result}
    \end{figure}
    
    
    
    A new hypothesis was formed. The team agreed that locally the pillars might be experiencing a high current density causing excessive heat which charred and damaged the resist mask. Parameters to the experiment were modified to reduce the current which increased the plating time. Once process parameters were adjusted the plating process was carried out again. 
    
    At the end of this attempt it was observed that the mask physically lifted off of the wafer and remained floating in the indium bath in flakes. Observing this phenomenon under the microscope showed what looked like violent damage to the resist mask. This caused portions of the mask to crack and lift off allowing \ce{In} to plate in regions of the wafer where the resist mask should prevent it. The team contacted the manufacturer of the bath and the chemicals to inquire about the compatibility of the mask. The response was that the mask, bath and chemicals should be compatible and that an over exposure and an additional bake should be done to to make the mask more rigid. However, this recommendation didn't solve the issues either. 
    
    It was at this point that the project scope was shifted to focus on the material analysis. The wafers were processed without any features on the front side. A bonding attempt was made with two non-patterned wafers, having only smooth metal layers. Results of this were inconclusive as only partial bond was achieved in center parts, while the wafers were not bonded near edges. Another attempt was made using chip level bonding, where square chips were diced out of non-patterned wafers and subjected to bonding with various parameters. Varied parameters were temperature, contact force and time. The most successful chip level bond was achieved in 250\celsius ~with 2500~s bonding time. In such case there was no pure tin left anymore and \ce{Cu6Sn5} was the dominant IMC.
    
    \subsection{Future Recommendations}
    
    The findings of this project were promising. This section will discuss the elements of the project that should be carried out in the future. The largest problem was the consistency of the \ce{In} plating. As such, corrective measures could be taken going forward. Primarily the indium plating process should be optimized and the incompatibilities experienced with the photoresist mask should be resolved. A hard mask or a new photoresist could be taken into use as possible solutions to this issue. In addition, it will be necessary to measure the thickness and surface consistency of the plated \ce{In} and perhaps whether it would be useful to plate tin on top of it. It may also be useful to consider attempting to deposit thicker layers of the low temperature materials to help control the electroplating process. 
    
    Despite having a successful bond with complete intermetallic formation at a temperature of 250\celsius, the parameters for making a successful wafer level bond should be adjusted. Currently, it is known that the system reaches a state where there is no more pure tin left at some bonding time between 500 and 2500 seconds. A natural next step would be to adjust the temperature and time parameters to create a bond of similar structure. Optimally, for the bonding machine, a time from 10 minutes up to 2 hours should be used. Additionally the bonding force parameter should be adjusted as during the chip level experiments the bonding pressure is an unknown. A spring rig was used instead of a weight to apply bonding force so not to increse the thermal mass significantly by adding a large copper weight.  
    
    The addition of the small amount of indium in this system appeared to stabilize the system and improve wetting properties. The issue was that there was somewhere in the range of 10\% of the desired indium in the system. It would be interesting to ascertain the effect (if any) on the reliability and strength of the bond. Additionally, (assuming the indium plating process is optimized) wafer level bonds with the desired amount of indium should be bonded, the bond characterized, and then tested for strength.  
    
    \clearpage
    
    \subsection{Reflection}
    
    In the project plan the team determined that using a traffic light system for determining the success of the project should be used. The project was graded on five factors. The factors were: 
    
    \begin{itemize}
        \item Schedule: Did the team meet all internal deadlines? 
        \item Project Scope: was a successful bond achieved at sub 300\celsius?
        \item Satisfaction of the team: Was the team satisfied with the project and their contribution?
        \item Customer Satisfaction: Were the instructor and advisor satisfied with the project?
        \item Quality of work: Did the team follow their iterative quality review process laid out in the project report?  
    \end{itemize}
    
    The team did not meet all of the internal deadlines, however the course deadlines were all met well within time. The largest factor for this was that the projected workloads for the project milestones were designed to meet the course's estimated workload. However, significantly more effort was required to maintain the quality standard of the team. In addition, the inexperience with respect to cleanroom processing and managing shared resources added additional time to the project. Lastly, some late additions and changes by the course administration caused some minor and unnecessary delays. Based on this criteria the result would be a yellow light.  
    
    A successful \ce{CuSnIn} bond was achieved at 250\celsius~ despite the difficulties with the indium plating process. Based on this factor, the team has assessed the project scope to have a green light. 
    
    The project exceeded the expectations of the team and as such the satisfaction is graded with a green light. In addition the project instructor and advisor were also pleased with the project. 
    
    The quality of the project has met the standards of the project group and portions of the project were completed individual and scrutinized through a rigorous peer review process. The quality of the work done within the scope of this project was marked with a green light.
    
    Based on the criteria set forth in the project report the overall success of this project is a green light. The main area where improvements could have been made was in the schedule of the project. However, since the project met all course deadlines, the effect of the delays overall was not catastrophic.   

\clearpage


\section{Conclusion}
    
    Based on the research completed within the scope of this project multiple metal systems were deemed to be potential solutions to the problem of achieving a low temperature bond. However, a \ce{CuSnIn} system was deemed to be the most suitable for this project, based on the material research. This is due to the well known behavior of \ce{CuSn} as a solder and the fact that adding Indium to the system should reduce the melting point while still forming stable IMCs.  
    
    Experimentally  \ce{CuSnIn} appears to be a promising material system for low temperature SLID bonding, as bonds were successfully carried out at 250\celsius~at chip level where a photo resist mask was not used to create features on the wafers. 
    
    In spite of the fact that the amount of indium introduced into the system was less than intended, the indium seems to stabilize the reactions of the system. With a temperature of 250\celsius~and a bonding time of 500 seconds a very small amount of pure \ce{Sn} remained in the bonded structure and at 2500 seconds there was no pure \ce{Sn} present in the structure at all. Information about the reliability of the bond is an aspect that should be investigated in the future.
     
\clearpage

\section{Personal Goal Fulfillment}

    Mikael's personal goal of deepening understanding of material science, especially interpreting IMCs from ternary phase diagrams was fulfilled. Some knowledge was also accrued about thin film deposition processes during wafer processing for experiments. Another personal goal was improving research skills, focusing on finding and tracking source materials, judging their trustworthiness and planning of experiments. This goal was fulfilled very well for source materials and their criticism, and well enough for experiment planning. Further personal development could be done on properties and application of different photo-resist materials.
    
    Joseph's personal goal of understanding the basic principles of material science was met and exceeded. The desire to gain experience in silicon processing was also achieved. Successfully completing cleanroom and chemical training in Aalto and gaining access to the processing facilities in Micronova was another successful desire. Group work, research and writing skills were also improved during this project. Lastly, there is realization that this type of research and work at Aalto is desirable, challenging, rewarding, interesting and he would really hope to continue to study and be involved in this type of work in the future.
    
    Joonas' personal goal of understanding the most commonly used bonding techniques and the basics of material science were met and exceeded. Acquired cleanroom working skills will be beneficial in the future. One goal was to improve research skills and especially technical writing. This goal was fulfilled very well and improvement is seen in each and every milestones during this project. Another personal goal was to improve project management and cooperation skills and this goal was achieved as well. In addition to acquired skills, this project has greatly increased his interest in this type of research. 

	
\clearpage	

%%%%%%%%%%%%%%%%% Reference Section Don not edit. Add sources to the Project_Work_Course.bib file %%%%%%%%%%%%%%%%%

%\begin{*thebibliography}%{99}

 %\nocite{*}

 \bibliographystyle{IEEEtran} 
 \bibliography{IEEEabrv,Project_Work_Course}

%\end{*thebibliography}


\clearpage

%%%%%%%%%%%%%%%%%%%%%%%%%%%%%%%%%%%%%%%%%% Appendices If needed %%%%%%%%%%%%%%%%%%%%%%%%%%%%%%%%%%%%%%%%%%%%%%%%%%%
\section{Appendices}
\appendix

%\iffalse
\section{SLID Bonding example process flow}
\label{App:Process}

%\centering
\includepdf[pages=-, offset=100 -60, frame=false]{Example_process.pdf}


\section{Indium electroplating microscope figures}
\label{App:In_Plating}
This appendix consist of set of indium electroplating optical microscope figures

\begin{figure}[H]
        \begin{center}
        \includegraphics[height=8cm]{appind1.jpg}
        \end{center}
        \caption{Example 1}
\end{figure}

\begin{figure}[H]
        \begin{center}
        \includegraphics[height=8cm]{appind2.jpg}
        \end{center}
        \caption{Example 2}
\end{figure}

\begin{figure}[H]
        \begin{center}
        \includegraphics[height=8cm]{appind3.jpg}
        \end{center}
        \caption{Example 3}
\end{figure}

\begin{figure}[H]
        \begin{center}
        \includegraphics[height=8cm]{appind4.jpg}
        \end{center}
        \caption{Example 4}
\end{figure}

\begin{figure}[H]
        \begin{center}
        \includegraphics[height=8cm]{appind5.jpg}
        \end{center}
        \caption{Example 5}
\end{figure}

\begin{figure}[H]
        \begin{center}
        \includegraphics[height=8cm]{appind6.jpg}
        \end{center}
        \caption{Example 6}
\end{figure}

\begin{figure}[H]
        \begin{center}
        \includegraphics[height=8cm]{appind7.jpg}
        \end{center}
        \caption{Example 7}
\end{figure}

\clearpage

\section{SEM figures of wafer bonded samples}
\label{App:SEM}
This appendix consist of a set of SEM figures of bonded samples. These figures are marked with points and areas which spectrum results are found in appendix \ref{App:SEM}. 

\begin{figure}[H]
        \begin{center}
        \includegraphics[height=8cm]{NWSEM2.jpg}
        \end{center}
        \caption{Another SEM figure of NW sample. Upper side of the bond consist mostly of tin and the side below the bond has majority of indium.}
\end{figure}

\begin{figure}[H]
        \begin{center}
        \includegraphics[height=8cm]{NESEM2.jpg}
        \end{center}
        \caption{Another SEM figure of NE sample. Upper side of the bond consist mostly of tin and the side below the bond has majority of indium.}
\end{figure}

\begin{figure}[H]
        \begin{center}
        \includegraphics[height=5cm]{Spectrum_P1.jpg}
        \end{center}
        \caption{P1 spectrum}
        \label{spectrum1}
\end{figure}

\begin{figure}[H]
        \begin{center}
        \includegraphics[height=5cm]{Spectrum_P2.jpg}
        \end{center}
        \caption{P1 spectrum}
        \label{spectrum2}
\end{figure}

\begin{figure}[H]
        \begin{center}
        \includegraphics[height=5cm]{Spectrum_P3.jpg}
        \end{center}
        \caption{P3 spectrum}
        \label{spectrum3}
\end{figure}

\begin{figure}[H]
        \begin{center}
        \includegraphics[height=5cm]{Spectrum_P4.jpg}
        \end{center}
        \caption{P4 spectrum}
        \label{spectrum4}
\end{figure}

\begin{figure}[H]
        \begin{center}
        \includegraphics[height=5cm]{Spectrum_A1.jpg}
        \end{center}
        \caption{A1 spectrum}
        \label{spectrum5}
\end{figure}

\begin{figure}[H]
        \begin{center}
        \includegraphics[height=8cm]{LINESCAN2.jpg}
        \end{center}
        \caption{Line scan from the bond of NE sample with all the elements}
        \label{LINE3}
\end{figure}


\begin{figure}[H]
        \begin{center}
        \includegraphics[height=8cm]{LINESCAN1andSEM2.jpg}
        \end{center}
        \caption{Line scan from the bond of NE sample with all the elements added into the SEM figure with scale}
        \label{LINE4}
\end{figure}

\begin{figure}[H]
        \begin{center}
        \includegraphics[height=8cm]{LINE5.jpg}
        \end{center}
        \caption{Additional SEM figure of chip level bond with scale. Bonding time 2500s}
        \label{LINE5}
\end{figure}

\begin{figure}[H]
        \begin{center}
        \includegraphics[height=8cm]{line6.jpg}
        \end{center}
        \caption{Line scan from previous figure}
        \label{LINE6}
\end{figure}

%\clearpage

%\includepdf[pages=-, offset=100 -60, frame=false]{P1.pdf}
%\includepdf[pages=-, offset=100 -60, frame=false]{P2.pdf}
%\includepdf[pages=-, offset=100 -60, frame=false]{P3.pdf}
%\includepdf[pages=-, offset=100 -60, frame=false]{P4.pdf}
%\includepdf[pages=-, offset=100 -60, frame=false]{A1.pdf}

\clearpage

\section{Material Matrix}
\label{App:Matrix}
%\includepdf[pages=-, offset=100 -60, frame=false]{Material_matrix.pdf}

\begin{figure}[H]
\includepdf[pages=1,pagecommand={},offset=2.5cm -5cm]{Material_matrix.pdf}
%\vspace{-6cm}\hspace{2cm}\includepdf[width=18cm]{Material_matrix.pdf}\hspace{-2cm}    
\end{figure}


%\fi
%\section*{Appendix 1: STM32F103 Datasheet}
%\appendix
%\centering
%\includepdf[pages=-, offset=75 -75, frame=true]{STM32F103.pdf}
%\includepdf[pages=-, offset=75 -60, frame=false]{NXP-LPC84x-Block-Diagram.pdf}
%\clearpage

%\section*{Appendix 2: Some Other Appendix}

%\begin{figure}[htb]
%\begin{center}
%\end{center}

%\includegraphics[height=10cm]{paretoafter}

%\caption*{miniFactory Innovator 3D Printer [1]}
%\label{liitekuva}
%\end{figure}

\end{document}

\fi
